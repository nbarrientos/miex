\chapter*{Agradecimientos}

Este proyecto ha sido �ntegramente desarrollado utilizando software libre, por
lo que en primer lugar me gustar�a agradecer a todas las personas que
han dedicado su tiempo a desarrollar estas maravillosas herramientas, en
gran parte, este proyecto tambi�n es suyo. Ser�a pr�cticamente imposible nombrar
una por una a todas ellas, por lo que voy a nombrar directamente los proyectos
que me han sido �tiles para llevar a cabo el m�o, sin ning�n orden en
particular:

\begin{quote}
\textit{Debian, GNU Make, Apache Ant, JSAP, MySQL Connector/J, SAX, Stanford
Parser, Vim, Umbrello, Texlive, CTAN, KDE, Kile, Dia, Kernel.org, amaroK, Mutt,
Procmail, Fetchmail, y muchos m�s...}
\end{quote}

Mi manera de dar las gracias es liberando este proyecto para, modestamente,
hacer que la comunidad del software libre siga creciendo.

Por otro lado, me gustar�a tambi�n dar las gracias a todas las personas que
hab�is, de una u otra forma, ayudado en el desarrollo de este proyecto revisando
la documentaci�n, haciendo pruebas o d�ndome ideas. 

Gracias a Elena y a Susana Irene por dirigir este proyecto. A Carlos D�ez y
Pablo Manuel de Roque por revisar la documentaci�n en busca de erratas. A Sergio
Fern�ndez, por permitirme inspirarme en la documentaci�n de su proyecto. Por
�ltimo, pero no por ello menos importante, a mi familia, quienes me han dado
siempre todo lo mejor y gracias a ellos hoy estoy escribiendo estas l�neas.

\begin{flushright}
\textit{El software, como el resto de la ciencia, libre.} -- Ren� M�rou
\begin{small}<\url{h@es.gnu.org}>\end{small}
\end{flushright}



