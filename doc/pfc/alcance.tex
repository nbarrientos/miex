\chapter*{Alcance de este documento}

Este libro contiene la documentaci�n del proyecto titulado \emph{\TituloPFC} y
pretende explicar lo m�s breve y concisamente posible como se ha desarrollado
as� como dar las claves necesarias para utilizarlo.

Adem�s de una peque�a memoria que abarcar� los aspectos no t�cnicos y el
historial del proceso de desarrollo, se incluye un cap�tulo completo destinado
a personal t�cnico (principalmente desarrolladores), un manual para
usuarios finales y un peque�o ep�grafe destinado a experimentaci�n.

\section*{Palabras clave}

Recuperaci�n de informaci�n, analizador sint�ctico, analizador sem�ntico,
Stanford Parser, SQL, Java.

\section*{Aclaraci�n sobre el t�tulo}

Aunque el t�tulo del proyecto escogido para la distribuci�n y utilizaci�n en el
marco universitario es \textit{\TituloPFC}, el t�tulo original
es \textbf{\TituloPFCWeb}, por lo que, a lo largo de este documento, se har�
referencia al proyecto utilizando los dos t�tulos indistintamente.

\section*{Historial}

\begin{center}
\begin{tabular}{rcl}
 \hline \hline
 \multicolumn{1}{c}{\textbf{Fecha}} &
 \multicolumn{1}{c}{\textbf{Versi�n}} &
 \multicolumn{1}{c}{\textbf{Comentarios}} \\
 \hline \hline
 25/06/07 & 0.1 & Primer borrador para revisi�n.\\
 \hline
 18/07/07 & 0.2 & Segundo borrador para revisi�n.\\
 \hline
 nn/07/07 & 1.0 & Primera versi�n para publicaci�n.\\
 \hline \hline
\end{tabular}
\end{center}