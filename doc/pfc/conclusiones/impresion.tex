\section{Impresiones personales}

Por la parte acad�mica, este proyecto me ha servido para tener un primer
contacto con el mundo de la recuperaci�n de informaci�n y el an�lisis de
lenguaje natural.

En los aspectos t�cnicos, el desarrollo me ha ayudado a practicar con Java, y a
refrescar mis conocimientos de \LaTeX. Aunque me he ayudado de un conjunto de
libros \cite{IntroLaTeX} \cite{LaTeXimprenta} y \cite{LearningJava}, la ayuda
m�s significativa la he obtenido de diversas p�ginas web, recogidas en
el ap�ndice \ref{cha:referencias}.

En lo personal, estando acostumbrado al trabajo en grupo en diversos proyectos,
me he dado cuenta de lo aburrido que puede ser trabajar uno s�lo a veces. Esta
falta de compa�eros no se nota tanto cuando se realizan tareas divertidas, como
el desarrollo del software, pero si a la hora de hacer tareas m�s pesadas, como
escribir documentaci�n.

No me extiendo m�s, si quieres saber m�s cosas si�ntete libre de preguntarme
por correo electr�nico\footnote{mailto:nacho@criptonita.com}, porque como se
suele decir\ldots

\begin{flushright}
\textit{Lo bueno, si es breve, doblemente bueno.} -- Popular
\end{flushright}

