\section{El futuro}

Como se ha comentado anteriormente, en la medida de lo posible, este proyecto se
continuar� manteniendo. Al ser un proyecto libre, el autor espera que pueda
ser utilizado por mucha m�s gente para este uso en concreto o para tomarlo como
base y modificarlo.

Si alg�n lector de este libro est� interesado en seguir mejorando MIEX aqu� se
recogen una serie de ampliaciones que pueden resultar interesantes.

\begin{itemize}
\item Desarrollo de una aplicaci�n web
utilizando AJAX\footnote{http://es.wikipedia.org/wiki/AJAX} y alg�n lenguaje
de uso frecuente en web (por
ejemplo PHP\footnote{http://es.wikipedia.org/wiki/PHP}) para la consulta
din�mica de la base de datos, donde el investigador pueda construir de una
manera sencilla las consultas.

\item A�adir un sistema para que en tiempo de ejecuci�n se pueda personalizar
el nombre de las etiquetas de los ficheros XML a tratar y as� evitar que
est� \textit{hardcoded} en el c�digo fuente.

\item Internacionalizaci�n del software, por ejemplo utilizando una biblioteca
con el mismo prop�sito que
GNU Gettext\footnote{http://www.gnu.org/software/gettext/} pero escrita en Java.

\item Sacando ventaja del anterior punto, localizar el programa a distintos
idiomas.

\end{itemize}
