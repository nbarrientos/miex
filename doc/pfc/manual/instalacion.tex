\section{Instalaci�n}

Una vez instaladas todas estas dependencias se proceder� a descargar el c�digo
fuente del programa. Para obtenerlo puede optar por cualquiera de estas
posibilidades, de las cuales se recomiendan las dos �ltimas, ya que,
probablemente, la versi�n que se distribuye en el CD-ROM quedar� pronto
desfasada.

\begin{itemize}
	\item Desde el CD-ROM que acompa�a a este tomo.
	\item Descarga de una \textit{release} desde la p�gina web del proyecto.
	\footnote{http://miex.sf.net/}
	\item Desde el repositorio de fuentes del proyecto.
\end{itemize}

Para las dos primeras opciones, es suficiente con descomprimir y desempaquetar
el \textit{tarball} de la siguiente manera (supongamos que estamos utilizando
la versi�n 0.1 del software).

\begin{small}
\texttt{\$ tar xzvf miex-0.1.tar.gz}	
\end{small}

En cambio, si prefiere decantarse por la tercera opci�n y obtener la �ltima
versi�n del c�digo deber� de instalar antes
Subversion\footnote{http://subversion.tigris.org/} y posteriormente ejecutar:

\begin{small}
\texttt{\$ svn co https://miex.svn.sf.net/svnroot/miex/trunk/miex miex}
\end{small}

Una vez que tengamos el c�digo, nos situamos en el directorio que contenga las
fuentes utilizando \textit{cd} y ejecutamos cualquiera de las siguientes
�rdenes, dependiendo de el sistema de construcci�n por el que se haya optado
anteriormente.

\begin{small}
\texttt{\$ make}	
\end{small}

O bien, si prefiere usar Ant,

\begin{small}
\texttt{\$ ant}	
\end{small}

Finalmente, para instalarlo en el sistema bajo el \textit{prefix} \texttt{/usr}
(es posible modificarlo en el fichero \textit{build.xml} o en el
\textit{Makefile}) es necesario teclear lo siguiente.

\begin{small}
\texttt{\$ sudo make install}	
\end{small}

O bien, si prefiere usar Ant,

\begin{small}
\texttt{\$ sudo ant install}	
\end{small}

Si no utiliza Sudo, simplemente haga que su UID efectivo sea 0, y ejecute la
misma orden sin precederla por \textit{sudo}.







