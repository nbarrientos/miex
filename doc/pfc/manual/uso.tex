\section{Utilizaci�n}

Antes de empezar a ver alg�n ejemplo de utilizaci�n, es necesario contar con un
sistema de gesti�n de bases de datos MySQL\footnote{http://www.mysql.com/}
instalado en alguna m�quina, bien sea la propia m�quina donde se vaya a ejecutar
MIEX o una m�quina remota.

La mayor�a de las distribuciones de GNU/Linux distribuyen MySQL mediante su
sistema de paqueter�a, por ejemplo en Debian GNU/Linux ser�a suficiente con
ejecutar como root:

\begin{small}
\texttt{\# sudo apt-get install mysql-server}	
\end{small}

Posteriormente utilice el manual de MySQL para crear una base de datos vac�a y
un usuario que tenga permisos de escritura en la misma. Si no se encuentra
c�modo con el cliente cl�sico de MySQL puede instalar alg�n gestor del estilo
de phpMyAdmin\footnote{http://www.phpmyadmin.net/}.

En este punto MIEX ya deber�a de estar correctamente instalado, puede
comprobarlo ejecut�ndolo sin ning�n argumento. Deber�a de obtener una salida
similar a esta:

\begin{Verbatim}[fontsize=\relsize{-1}]
$ miex
Error: Parameter 'configFileURI' is required.
Error: Parameter 'files' is required.

Usage: ./run
                (-c|--config) <confFile.conf> files1 files2 ... filesN
...
\end{Verbatim}

Antes de comenzar a analizar documentos, es necesario crear un fichero de
configuraci�n para ajustar el comportamiento del programa a los resultados que
se quieren obtener. Lo m�s sencillo es tomar como plantilla el que se encuentra
en \url{/usr/share/doc/miex/examples} y hacer las mofidificaciones pertinentes.

Los valores por omisi�n deber�an de ser adecuados por lo que s�lo deber�a de
rellenar las claves que configuran las credenciales para acceder a la base de
datos y la clave \textit{CreateDB} que, al menos la primera vez, deber�a de
estar a \texttt{Yes}.

En el fichero de configuraci�n de ejemplo, en forma de comentarios, se explica
la utilidad de cada una de las opciones por lo que no se volver�n a explicar c
ada una de ellas de nuevo en este cap�tulo.

Como ejemplo, vamos a analizar el fichero de ejemplo que se distribuye al
instalar el programa, localizado en \url{/usr/share/doc/miex/examples/input/}

Suponiendo que el fichero de configuraci�n est� situado en el directorio
actual, procedamos a la ejecuci�n.

\begin{small}
\texttt{\$ miex -c example.conf /usr/share/doc/miex/examples/input/example.xml}	
\end{small}

N�tese que, por seguridad, antes de comenzar el an�lisis es necesario que usted
confirme que quiere borrar el contenido actual de la base de datos (en el caso
de que existiera).

Como el programa es bastante \textit{verboso} se puede ver f�cilmente como
procesa cada colecci�n y cada documento. Separa el t�tulo del cuerpo y analiza
tanto las dependencias como las propiedades de cada palabra por separado. A
continuaci�n se explica el significado de cada una de las letras que muestran la
evoluci�n del proceso a nivel de oraci�n.

\begin{itemize}
	\item \textbf{C} - La oraci�n est� \textit{cacheada} en el
\textit{parser}.
	\item \textbf{R(n)} - Se han eliminado n elementos in�tiles.
	\item \textbf{F} - Se ha aplicado el filtro.
	\item \textbf{S} - Se ha ejecutado una sentencia SQL.
	\item \textbf{N} - Se ha aplicado el algoritmo de normalizaci�n.
\end{itemize}

Una vez procesadas las colecciones que necesite, puede construir las consultas
que considere oportunas para obtener los datos que desee de la base de datos.


