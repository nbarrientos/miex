\section{Utilizaci�n}

En primer lugar utilice el manual de MySQL para crear una base de datos vac�a y
un usuario que tenga permisos de escritura en la misma. Si no se encuentra
c�modo con el cliente cl�sico de MySQL puede instalar alg�n gestor del estilo
de phpMyAdmin\footnote{http://www.phpmyadmin.net/}.

En este punto MIEX ya deber�a de estar correctamente instalado, puede
comprobarlo ejecut�ndolo sin ning�n argumento. Deber�a de obtener una salida
similar a esta:

\begin{Verbatim}[frame=single,framesep=10pt,fontsize=\relsize{-1}]
$ miex
Error: Parameter 'configFileURI' is required.
Error: Parameter 'files' is required.

Usage: ./run
                (-c|--config) <confFile.conf> files1 files2 ... filesN
...
\end{Verbatim}

Antes de comenzar a analizar documentos, es necesario crear un fichero de
configuraci�n para ajustar el comportamiento del programa a los resultados que
se quieren obtener. Lo m�s sencillo es tomar como plantilla el que se encuentra
en \url{/usr/share/doc/miex/examples} y hacer las mofidificaciones pertinentes.

Los valores por omisi�n deber�an de ser adecuados por lo que s�lo deber�a de
rellenar las claves que configuran las credenciales para acceder a la base de
datos y la clave \textit{CreateDB} que, al menos la primera vez, deber�a de
estar a \texttt{Yes}.

En el fichero de configuraci�n de ejemplo, en forma de comentarios, se explica
la utilidad de cada una de las opciones por lo que no se volver�n a explicar
cada una de ellas de nuevo en este cap�tulo.

Como ejemplo de fichero de colecci�n, vamos a analizar el fichero de ejemplo que
se distribuye al instalar el programa, localizado en
\url{/usr/share/doc/miex/examples/input/}

Suponiendo que el fichero de configuraci�n que hemos preparado (llamado
\textit{example.conf}) est� situado en el directorio actual, procedamos a la
ejecuci�n.

\begin{small}
\texttt{\$ miex -c example.conf /usr/share/doc/miex/examples/input/example.xml}	
\end{small}

N�tese que, por seguridad, si la clave \textit{CreateDB} est� a \textit{Yes},
antes de comenzar el an�lisis es necesario que usted confirme que quiere borrar
el contenido actual de la base de datos (en el caso de que existiera).

Como se puede comprobar en la figura \ref{fig:ejemplo-procesamiento}, los
mensajes que se imprimen por la salida est�ndar son bastante explicativos por lo
que se puede ver f�cilmente el proceso de an�lisis de cada uno de los documentos
que conforman las colecciones. Fij�ndose un poco m�s, tambi�n se puede apreciar
que el an�lisis de cada oraci�n se realiza en pasos descritos por una serie de
letras de las cuales se detalla su significado a continuaci�n.

\begin{figure}[h]
\centering
\begin{Verbatim}[frame=single,framesep=5pt,fontsize=\relsize{-1}]
Cmdline syntax is OK, continuing...

Config file is OK, continuing...

WARNING: Existing database will be completely deleted. [Y/N]: Y

Creating DB...

Parsing file "/usr/share/doc/miex/examples/input/example.xml"

Loading parser from serialized file lib/grammars/englishPCFG.ser.gz ... 
done [2.4 sec].

        Processing TITLE of the document titled: Debian project
                Processing sentence #1
                        Dependencies... [ R(0)  F  S ]
                        Properties...  [ C  R(0)  F  S ]

        Processing BODY of the document titled: Debian project
                Processing sentence #1
                        Dependencies... [ R(9)  F  S ]
                        Properties...  [ C  R(9)  F  S ]
                Processing sentence #2
                        Dependencies... [ R(11)  F  S ]
                        Properties...  [ C  R(10)  F  S ]
                Processing sentence #3
                        Dependencies... [ R(18)  F  S ]
                        Properties...  [ C  R(19)  F  S ]

        Processing TITLE of the document titled: Google is great.
                Processing sentence #1
                        Dependencies... [ R(1)  F  S ]
                        Properties...  [ C  R(2)  F  S ]

        Processing BODY of the document titled: Google is great.
                Processing sentence #1
                        Dependencies... [ R(6)  F  S ]
                        Properties...  [ C  R(11)  F  S ]
                Processing sentence #2
                        Dependencies... [ R(15)  F  S ]
                        Properties...  [ C  R(15)  F  S ]
\end{Verbatim}
\label{fig:ejemplo-procesamiento}
\caption{Informaci�n mostrada durante un procesamiento}
\end{figure}

\begin{itemize}
	\item \textbf{C} - La oraci�n est� \textit{cacheada} en el
\textit{parser}.
	\item \textbf{R(n)} - Se han eliminado n elementos in�tiles.
	\item \textbf{F} - Se ha aplicado el filtro.
	\item \textbf{S} - Se ha ejecutado una sentencia SQL.
	\item \textbf{N} - Se ha aplicado el algoritmo de normalizaci�n.
\end{itemize}

Una vez procesadas las colecciones que necesite, puede construir las consultas
que considere oportunas para obtener los datos que desee de la base de datos.


