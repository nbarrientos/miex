
\section{Requisitos software}

Antes de comenzar la instalaci�n, es necesario descargar y colocar en
\textit{/usr/lib/java} todos los JARs de las dependencias de compilaci�n y
ejecuci�n de MIEX. La siguiente lista refleja cuales son esas dependencias, las
 versiones recomendadas as� como donde se pueden obtener.

\begin{itemize}
	\item Java SE (JRE+JDK)
	\footnote{http://java.sun.com/javase/}
	\begin{small}(\textgreater= 5.0)\end{small}

	\item Stanford Parser
	\footnote{http://nlp.stanford.edu/} 
	\begin{small}(\textgreater= 2006-06-11)\end{small}
	
	\item JSAP
	\footnote{http://www.martiansoftware.com/jsap/}
	\begin{small}(\textgreater= 2.1)\end{small}
	 
	\item MySQL Connector/J
        \footnote{http://www.mysql.com/products/connector/j/}
	\begin{small}(\textgreater= 5.0.5)\end{small}
	 
	\item SAX
	\footnote{http://www.saxproject.org/}
	\begin{small}(\textgreater= 2.0)\end{small}
\end{itemize}

Para compilar e instalar es necesario una de las siguientes herramientas.

\begin{itemize}
	\item Apache Ant
	\footnote{http://ant.apache.org/}
	\item GNU Make
	\footnote{http://www.gnu.org/software/make/}
\end{itemize}

Finalmente, como sistema de gesti�n de bases de datos deber� de instalar
(en la misma m�quina o en una remota) MySQL\footnote{http://www.mysql.org} 5.0 o
superior.

La mayor�a de las distribuciones de GNU/Linux distribuyen MySQL mediante su
sistema de paqueter�a, por ejemplo en Debian GNU/Linux ser�a suficiente con
ejecutar como \textit{root}:

\begin{small}
\texttt{\# sudo apt-get install mysql-server}	
\end{small}