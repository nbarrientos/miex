\section{Requisitos hardware}

Como se advierte en la documentaci�n del Stanford Parser el coste computacional
del an�lisis es muy alto, por lo que es recomendable disponer de una m�quina de
prestaciones medias/altas para obtener resultados en un tiempo aceptable.

Por falta obvia de recursos, no se ha podido hacer pruebas en diferentes
configuraciones hardware por lo que, exceptuando el sistema operativo, t�mense
los siguientes requisitos como una recomendaci�n.

\begin{itemize}

\item \underline{Arquitectura}: Cualquier arquitectura es v�lida,
mientras exista una m�quina virtual de Java para ella (de Sun Microsystems o
compatible con la versi�n establecida en los requisitos software) y un sistema
operativo (recomendado GNU/Linux) que la soporte. Una elecci�n aceptable ser�a
tomar una de las m�s populares, por ejemplo x86-32 (IA-32) o x86-64 (EMT64).

\item \underline{Procesador}: Cualquier procesador que implemente una de las
arquitecturas v�lidas. La frecuencia de funcionamiento y la cantidad de cach�,
dentro de unos m�rgenes l�gicos, a convenir.

\item \underline{Memoria}: El consumo de memoria por parte del an�lisis de las
oraciones es bastante elevado, por lo que para no tener que hacer demasiado uso
de la memoria de intercambio, se recomiendan cantidades de memoria f�sica
iguales o superiores a 1 GiB.

\item \underline{Disco}: A convenir (si va a instalar la base de datos en el
equipo local, evalue el n�mero de colecciones a analizar para elegir una
capacidad adecuada).

\item \underline{Sistema Operativo}: Se recomienda cualquier distribuci�n de
\textbf{GNU/Linux}, aunque cualquier sistema operativo sobre el se pueda correr
una m�quina virtual de Java podr�a servir.

\item \underline{Otros}: A convenir.

\end{itemize}



