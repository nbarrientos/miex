\section{Almacenamiento de la informaci�n}

A la hora de decidir el medio para almacenar los resultados obtenidos
r�pidamente se descart� la opci�n de utilizar ficheros de texto plano, ya que
requerir�an la construcci�n de analizadores adiccionales para, una vez se
quiera consultar los resultados, poder recuperar la informaci�n.

Finalmente se decidi� utilizar una base de datos relacional para poder, en
cierta manera, dise�ar una estructura lo m�s c�moda posible para dejar abierto
un gran abanico de posibilidades a la hora de consultar la informaci�n obtenida.

En el cap�tulo que se dedica a los aspectos t�cnicos, se explica la base de
datos utilizada as� como la estructura de la misma. 