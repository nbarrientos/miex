\section{Recursos sobre el proyecto}

\subsection{P�gina web}

Una vez sea presentado el proyecto mi intenci�n es seguir desarrollando
MIEX en mis ratos libres. Para hacer el trabajo m�s transparente, existe un
proyecto creado en SourceForge\footnote{http://www.sourceforge.net} as� como una
p�gina web en la siguiente direcci�n.

\url{http://miex.sf.net}

A trav�s de esa URI es posible obtener la informaci�n m�s actualizada acerca de
MIEX, estar a d�a de los eventos que rodean al proyecto y obtener
\textit{snapshots} actualizadas del c�digo fuente.

\subsection{RSS}

Si desea suscribirse al canal RSS\footnote{http://es.wikipedia.org/wiki/RSS} que
contiene las noticias no tiene m�s que a�adir la siguiente direcci�n en su
agregador favorito:

\url{http://sourceforge.net/export/rss2_projnews.php?group_id=187602}

\subsection{Subversion}

El repositorio donde se lleva el control de las versiones del c�digo fuente
(utilizando Subversion\footnote{http://subversion.tigris.org/}) se
encuentra en \url{https://miex.svn.sourceforge.net/svnroot/miex} y tiene la
siguiente estructura:

\begin{itemize}
	\item \textbf{branches} - Ramas de desarrollo paralelas a
\textit{trunk}
	\item \textbf{tags} - Versiones congeladas (releases)
	\item \textbf{trunk} - Rama principal de desarrollo de MIEX 
	\begin{itemize}
		\item \textbf{doc} - Documentaci�n (presentaciones, libros...)
		\item \textbf{miex} - C�digo fuente del proyecto
		\item \textbf{web} - P�gina web
	\end{itemize}
\end{itemize}

Use el contenido de \textit{trunk} con precauci�n, debido a que es c�digo que
pueda cambiar con mucha frecuencia, puede obtener una captura inestable y no
obtener los resultados deseados. Si desea una versi�n estable y funcional
desc�rguela directamente de la p�gina web o obt�ngala de \textit{tags}.

