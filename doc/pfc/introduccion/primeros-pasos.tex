\section{Primeros pasos}

La importancia que posee la recuperaci�n de informaci�n es vital en nuestros
d�as. A partir de un cierto volumen de texto se hace imprescindible un
sistema organizativo que posibilite la localizaci�n de la informaci�n que se
precise en cualquier momento. Como es de suponer, para construir los
citados sistemas organizativos es necesario el an�lisis previo, tanto sint�ctico
como sem�ntico, de los textos.

En general, los analizadores de lenguaje natural existentes tienen varias
desventajas. Una de ellas, es que no poseen interfaces que hagan f�cil su manejo
y por tanto hacen perder mucho tiempo al usuario en escribir c�digo adicional o
en leerse mucha documentaci�n para utilizarlos. Por otro lado, no suelen
proporcionar mecanismos para guardar la informaci�n extra�da en medios flexibles
para posteriores consultas.

El objetivo de este proyecto es la elaboraci�n de un envoltorio
(\textit{wrapper}) de uno de los analizadores m�s conocidos, el Stanford
Parser. Con �ste envoltorio se pretende hacer que el manejo del mismo sea lo
m�s sencillo posible, para que, la persona que lo use, se pueda abstraer
totalmente del funcionamiento interno del software y poder analizar documentos
f�cilmente. Para paliar la segunda desventaja de estos analizadores, se
proporcionar� un sistema para el almacenamiento de la informaci�n en una base
de datos relacional.

Es importante que el usuario de este proyecto tenga en cuenta que las
capacidades de an�lisis tanto sem�ntico como sint�ctico est�n restringidas a las
propias limitaciones del analizador base que se toma, por lo que se aconseja
encarecidamente leer documentaci�n sobre el Stanford Parser antes de continuar
con este libro.

Si lo que quiere es aprender directamente como utilizar MIEX, le aconsejo que
lea directamente al cap�tulo \ref{cap:manual}, donde se explica como ponerlo
en marcha. Si prefiere conocer el formato de las colecciones a analizar o la
informaci�n que se extrae de las mismas, siga leyendo este ep�grafe.




