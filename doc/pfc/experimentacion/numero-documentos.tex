\section{An�lisis respecto al n�mero de documentos}

En el an�lisis de las colecciones, no es tan dif�cil predecir el tiempo que
puede tardar en completarse el an�lisis si s�lo tenemos en cuenta como variable
el n�mero de documentos a analizar. Como veremos en la siguiente secci�n existe
otra cuesti�n a tener en cuenta mucho m�s importante a la hora de medir lo que
puede llevar procesar una serie de entradas. 

Para realizar este estudio se ha creado un documento y se ha utilizado para
construir colecciones con 10, 25, 50, 100 y 500 documentos iguales. Como es de
esperar, el tiempo de procesamiento crece aproximadamente de manera lineal en
funci�n del n�mero de documentos, como se puede observar en la tabla
\ref{tab:longitud-colecciones}.

\begin{table}
\begin{center}
\begin{tabular}{ccc}
 \hline \hline
 \multicolumn{1}{c}{\textbf{Documentos}} &
 \multicolumn{1}{c}{\textbf{Tiempo \small{(segundos)}}} &
 \multicolumn{1}{c}{\textbf{Tiempo formateado \small{(aprox)}}} \\
 \hline \hline
 10 & 31,488 & 0m 31s\\
 \hline
 25 & 69,708 & 1m 9s\\
 \hline
 50 & 133,932 & 2m 13s\\
 \hline
 100 & 276,026 & 4m 36s\\
 \hline
 500 & 1271,352 & 21m 11s\\
 \hline \hline
\end{tabular}
\caption{Tiempo de procesamiento en funci�n del n�mero de documentos}
\label{tab:longitud-colecciones}
\end{center}
\end{table}
