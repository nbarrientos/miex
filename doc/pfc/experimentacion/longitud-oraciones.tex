\section{An�lisis respecto a la longitud de las oraciones}

Para esta prueba, se han construido una serie de colecciones con un s�lo
documento formado por dieciocho oraciones. Cada una de esas colecciones
var�a el tama�o medio de las oraciones que componen el documento que hay en su
interior. En la tabla \ref{tab:longitud-oraciones} se presentan los resultados
obtenidos.

\begin{table}
\begin{center}
\begin{tabular}{cc}
 \hline \hline
 \multicolumn{1}{c}{\textbf{Tama�o medio \small{(oraciones)}}} &
 \multicolumn{1}{c}{\textbf{Tiempo \small{(segundos)}}} \\
 \hline \hline
 5 & 12,44\\
 \hline
 10 & 15,03\\
 \hline
 \textbf{15} & \textbf{18,15}\\
 \hline
 20 & 28,45\\
 \hline
 25 & 43,56\\
 \hline
 \textbf{30} & \textbf{53,22}\\
 \hline \hline
\end{tabular}
\caption{Tiempo de procesamiento en funci�n de la longitud de las oraciones}
\label{tab:longitud-oraciones}
\end{center}
\end{table}

Como se puede apreciar en el caso resaltado, duplicar el tama�o medio de
las oraciones casi triplica el tiempo de procesamiento.
