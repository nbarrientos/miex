\chapter{English Grammatical Relationships\label{cha:gramrelations}}

En este ep�grafe se recogen las etiquetas utilizadas para describir las
relaciones entre las palabras. Si quiere profundizar en el tema, puede echar un
vistazo a \cite{TypedDependency}.

\begin{center}
\begin{longtable}{rcl}

% Cabecera principal
\hline \hline \\[-2ex]
   \multicolumn{1}{c}{\textbf{N�mero}} &
   \multicolumn{1}{c}{\textbf{Etiqueta}} &
   \multicolumn{1}{c}{\textbf{Descripci�n}} \\%[0.5ex] 
   \hline
   \\[-2.6ex]
\endfirsthead

% Cabecera de las otras partes de la tabla
\multicolumn{3}{c}{{Continuaci�n del \tablename} \thetable{}}\\[1.5ex]
  \hline \hline \\[-2ex]
  \multicolumn{1}{c}{\textbf{N�mero}} &
  \multicolumn{1}{c}{\textbf{Etiqueta}} &
  \multicolumn{1}{c}{\textbf{Descripci�n}} \\
  \\[-2.8ex]
\endhead

% Pi� de las p�ginas, excepto de la �ltima
  \multicolumn{3}{l}{{Contin�a en la siguiente p�gina\ldots}} \\
\endfoot

% Pi� de la �ltima p�gina
\endlastfoot

% Datos
1 & ABBREV & Abbreviation appositional modifier\\
\hline
2 & ACOMP & Adjectival complement\\
\hline
3 & AMOD & Adjectival modifier\\
\hline
4 & ADVCL & Adverbial clause modifier\\
\hline
5 & ADVMOD & Adverbial modifier\\
\hline
6 & AGENT & Agent\\
\hline
7 & APPOS & Appositional modifier\\
\hline
8 & ARG & Argument\\
\hline
9 & AUX & Auxiliary\\
\hline
10 & AUXPASS & Passive auxiliary\\
\hline
11 & CCOMP & Clausal complement\\
\hline
12 & CSUBJ & Clausal subject\\
\hline
13 & COMP & Complement\\
\hline
14 & COMPLM & Complementizer\\
\hline
15 & CONJ & Conjunct\\
\hline
16 & XSUBJ & Controlling subject\\
\hline
17 & CC & Coordination\\
\hline
18 & COP & Copula\\
\hline
19 & DET & Determiner\\
\hline
20 & DOBJ & Direct object\\
\hline
21 & EXPL & Expletive\\
\hline
22 & IOBJ & Indirect object\\
\hline
23 & INFMOD & Infinitival modifier\\
\hline
24 & MARK & Marker\\
\hline
25 & MOD & Modifier\\
\hline
26 & NEG & Negation modifier\\
\hline
27 & NSUBJPASS & Nominal passive subject\\
\hline
28 & NSUBJ & Nominal subject\\
\hline
29 & NN & Noun compound modifier\\
\hline
30 & NUMBER & Compound number modifier\\
\hline
31 & NUM & Numeric modifier\\
\hline
32 & OBJ & Object\\
\hline
33 & PARTMOD & Participial modifier\\
\hline
34 & PRT & Phrasal verb particle\\
\hline
35 & POSS & Possession\\
\hline
36 & POSSESSIVE & Possessive\\
\hline
37 & PRECONJ & Preconjunct\\
\hline
38 & PREDET & Predeterminer\\
\hline
39 & PRED & Predicate\\
\hline
40 & PREP & Prepositional modifier\\
\hline
41 & POBJ & Prepositional object\\
\hline
42 & PUNCT & Punctuation\\
\hline
43 & PURPCL & Purpose clause modifier\\
\hline
44 & REF & Referent\\
\hline
45 & REL & Relative\\
\hline
46 & RCMOD & Relative clause modifier\\
\hline
47 & SUBJ & Subject\\
\hline
48 & TMOD & Temporal modifier\\
\hline
49 & XCOMP & Xclausal complement\\
\hline \hline
\caption{Etiquetas que relacionan las palabras}

\end{longtable}
\end{center}