\chapter{Penn Treebank II Tags\label{sec:penntags}}

En el siguiente anexo se explica en forma de tabla el significado de cada una
de las etiquetas que se puedan ir asignando durante el an�lisis de las
oraciones.
 

\begin{center}

\begin{longtable}{rcl}

% Cabecera principal
\hline \hline \\[-2ex]
   \multicolumn{1}{c}{\textbf{N�mero}} &
   \multicolumn{1}{c}{\textbf{Etiqueta}} &
   \multicolumn{1}{c}{\textbf{Descripci�n}} \\%[0.5ex] 
   \hline
   \\[-2.6ex]
\endfirsthead

% Cabecera de las otras partes de la tabla
\multicolumn{3}{c}{{Continuaci�n del \tablename} \thetable{}}\\[1.5ex]
  \hline \hline \\[-2ex]
  \multicolumn{1}{c}{\textbf{N�mero}} &
  \multicolumn{1}{c}{\textbf{Etiqueta}} &
  \multicolumn{1}{c}{\textbf{Descripci�n}} \\
  \\[-2.8ex]
\endhead

% Pi� de las p�ginas, excepto de la �ltima
  \multicolumn{3}{l}{{Contin�a en la siguiente p�gina\ldots}} \\
\endfoot

% Pi� de la �ltima p�gina
\endlastfoot

% Datos
1 & S & simple declarative clause, i.e. one that is not introduced \\
  &   & by a (possible empty) subordinating conjunction or a\\
  &   & \textit{wh}-word and that does not exhibit subject-verb inversion\\
\hline
2 & SBAR & Clause introduced by a (possibly empty) subordinating conjunction.\\
\hline
3 & SBARQ & Direct question introduced by a \textit{wh}-word or \\
  &       & a \textit{wh}-phrase.  Indirect questions and relative clauses\\
  &       & should be bracketed as SBAR, not SBARQ.\\
\hline  
4 & SINV & Inverted declarative sentence, i.e. one in which the\\
  &      & subject follows the tensed verb or modal.\\
\hline  
5 & SQ   & Inverted yes/no question, or main clause of a \textit{wh}-question,\\
  &      & following the \textit{wh}-phrase in SBARQ.\\
\hline \hline

\caption{Etiquetas a nivel de cl�usula}
\end{longtable}


%%%%%%%%%%%
%%%%%%%%%%%

\begin{longtable}{rcl}

% Cabecera principal
\hline \hline \\[-2ex]
   \multicolumn{1}{c}{\textbf{N�mero}} &
   \multicolumn{1}{c}{\textbf{Etiqueta}} &
   \multicolumn{1}{c}{\textbf{Descripci�n}} \\%[0.5ex] 
   \hline
   \\[-2.6ex]
\endfirsthead

% Cabecera de las otras partes de la tabla
\multicolumn{3}{c}{{Continuaci�n del \tablename} \thetable{}}\\[1.5ex]
  \hline \hline \\[-2ex]
  \multicolumn{1}{c}{\textbf{N�mero}} &
  \multicolumn{1}{c}{\textbf{Etiqueta}} &
  \multicolumn{1}{c}{\textbf{Descripci�n}} \\
  \\[-2.8ex]
\endhead

% Pi� de las p�ginas, excepto de la �ltima
  \multicolumn{3}{l}{{Contin�a en la siguiente p�gina\ldots}} \\
\endfoot

% Pi� de la �ltima p�gina
\endlastfoot

% Datos
1 & ADJP & Adjective Phrase\\
\hline
2 & ADVP & Adverb Phrase\\
\hline
3 & CONJP & Conjunction Phrase\\
\hline
4 & FRAG & Fragment\\
\hline
5 & INTJ & Interjection. Corresponds approximately to the part-of-speech\\
  &      & tag UH\\
\hline
6 & LST & List marker. Includes surrounding punctuation\\
\hline
7 & NAC & Not a Constituent; used to show the scope of certain prenominal\\
  &     & modifiers within an NP\\
\hline
8 & NP & Noun Phrase\\
\hline
9 & NX & Used within certain complex NPs to mark the head of the NP\\
\hline
10 & PP & Prepositional Phrase\\
\hline
11 & PRN & Parenthetical\\
\hline
12 & PRT & Particle. Category for words that should be tagged RP\\
\hline
13 & QP & Quantifier Phrase\\
\hline
14 & RRC & Reduced Relative Clause\\
\hline
15 & UCP & Unlike Coordinated Phrase\\
\hline
16 & VP & Vereb Phrase\\
\hline
17 & WHADJP & \textit{Wh}-adjective Phrase\\
\hline
18 & WHAVP & \textit{Wh}-adverb Phrase\\
\hline
19 & WHNP & \textit{Wh}-noun Phrase\\
\hline
20 & WHPP & \textit{Wh}-prepositional Phrase.\\
\hline
21 & X & Unknown, uncertain, or unbracketable\\
\hline \hline

\caption{Etiquetas a nivel de frase}

\end{longtable}


%%%%%%%%%%%%%
%%%%%%%%%%%%%


\begin{longtable}{rcl}

% Cabecera principal
\hline \hline \\[-2ex]
   \multicolumn{1}{c}{\textbf{N�mero}} &
   \multicolumn{1}{c}{\textbf{Etiqueta}} &
   \multicolumn{1}{c}{\textbf{Descripci�n}} \\%[0.5ex] 
   \hline
   \\[-2.6ex]
\endfirsthead

% Cabecera de las otras partes de la tabla
\multicolumn{3}{c}{{Continuaci�n del \tablename} \thetable{}}\\[1.5ex]
  \hline \hline \\[-2ex]
  \multicolumn{1}{c}{\textbf{N�mero}} &
  \multicolumn{1}{c}{\textbf{Etiqueta}} &
  \multicolumn{1}{c}{\textbf{Descripci�n}} \\
  \\[-2.8ex]
\endhead

% Pi� de las p�ginas, excepto de la �ltima
  \multicolumn{3}{l}{{Contin�a en la siguiente p�gina\ldots}} \\
\endfoot

% Pi� de la �ltima p�gina
\endlastfoot

% Datos
1 & CC & Coordinating conjunction\\
\hline
2 & CD & Cardinal number\\
\hline
3 & CC & Coordinating conjunction\\
\hline
4 & CD & Cardinal number\\
\hline
5 & DT & Determiner\\
\hline
6 & EX & Existential there\\
\hline
7 & FW & Foreign word\\
\hline
8 & IN & Preposition or subordinating conjunction\\
\hline
9 & JJ & Adjective\\
\hline
10 & JJR & Adjective, comparative\\
\hline
11 & JJS & Adjective, superlative\\
\hline
12 & LS & List item marker\\
\hline
13 & MD & Modal\\
\hline
14 & NN & Noun, singular or mass\\
\hline
15 & NNS & Noun, plural\\
\hline
16 & NNP & Proper noun, singular\\
\hline
17 & NNPS & Proper noun, plural\\
\hline
18 & PDT & Predeterminer\\
\hline
19 & POS & Possessive ending\\
\hline
20 & PRP & Personal pronoun\\
\hline
21 & PRP\$ & Possessive pronoun (prolog version PRP-S)\\
\hline
22 & RB & Adverb\\
\hline
23 & RBR & Adverb, comparative\\
\hline
24 & RBS & Adverb, superlative\\
\hline
25 & RP & Particle\\
\hline
26 & SYM & Symbol\\
\hline
27 & TO & to\\
\hline
28 & UH & Interjection\\
\hline
29 & VB & Verb, base form\\
\hline
30 & VBD & Verb, past tense\\
\hline
31 & VBG & Verb, gerund or present participle\\
\hline
32 & VBN & Verb, past participle\\
\hline
33 & VBP & Verb, non-3rd person singular present\\
\hline
34 & VBZ & Verb, 3rd person singular present\\
\hline
35 & WDT & Wh-determiner\\
\hline
36 & WP & Wh-pronoun\\
\hline
37 & WP\$ & Possessive wh-pronoun (prolog version WP-S)\\
\hline
38 & WRB & Wh-adverb\\
\hline \hline

\caption{Etiquetas a nivel de palabra}

\end{longtable}
\end{center}
