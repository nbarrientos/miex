\chapter{Experimentaci�n}

La experimentaci�n realizada con el proyecto ha sido casi continua desde que
comenz� su desarrollo, incluso, como se comenta en anteriores cap�tulos, se han
construido herramientas auxilires para analizar de una manera m�s f�cil la
informaci�n obtenida.

En este cap�tulo se ha decidido recoger un peque�o an�lisis de rendimiento,
viendo cuales son los factores que afectan en mayor medida al tiempo de
procesamiento. Por si el lector siente especial curiosidad, estas pruebas se
han realizado en un \textit{AMD Athlon\texttrademark{} 64 X2 Dual Core
Processor 3800+} corriendo Debian GNU/Linux.

El tiempo de procesamiento depende de el n�mero de documentos a tratar y,
en mayor medida, de la longitud media de las oraciones de los mismos. En las
pr�ximas secciones, se presentan sendos an�lisis desde los dos puntos de vista.

\section{An�lisis respecto al n�mero de documentos}

En el an�lisis de las colecciones, no es tan dif�cil predecir el tiempo que
puede tardar en completarse el an�lisis si s�lo tenemos en cuenta como variable
el n�mero de documentos a analizar. Como veremos en la siguiente secci�n existe
otra cuesti�n a tener en cuenta mucho m�s importante a la hora de medir lo que
puede llevar procesar una serie de entradas. 

Para realizar este estudio se ha creado un documento y se ha utilizado para
construir colecciones con 10, 25, 50, 100 y 500 documentos iguales. Como es de
esperar, el tiempo de procesamiento crece aproximadamente de manera lineal en
funci�n del n�mero de documentos, como se puede observar en la tabla
\ref{tab:longitud-colecciones}.

\begin{table}
\begin{center}
\begin{tabular}{ccc}
 \hline \hline
 \multicolumn{1}{c}{\textbf{Documentos}} &
 \multicolumn{1}{c}{\textbf{Tiempo \small{(segundos)}}} &
 \multicolumn{1}{c}{\textbf{Tiempo formateado \small{(aprox)}}} \\
 \hline \hline
 10 & 31,488 & 0m 31s\\
 \hline
 25 & 69,708 & 1m 9s\\
 \hline
 50 & 133,932 & 2m 13s\\
 \hline
 100 & 276,026 & 4m 36s\\
 \hline
 500 & 1271,352 & 21m 11s\\
 \hline \hline
\end{tabular}
\caption{Tiempo de procesamiento en funci�n del n�mero de documentos}
\label{tab:longitud-colecciones}
\end{center}
\end{table}

\section{An�lisis respecto a la longitud de las oraciones}

Para esta prueba, se han construido una serie de colecciones con un s�lo
documento formado por dieciocho oraciones. Cada una de esas colecciones
var�a el tama�o medio de las oraciones que componen el documento que hay en su
interior. En la tabla \ref{tab:longitud-oraciones} se presentan los resultados
obtenidos.

\begin{table}
\begin{center}
\begin{tabular}{cc}
 \hline \hline
 \multicolumn{1}{c}{\textbf{Tama�o medio \small{(oraciones)}}} &
 \multicolumn{1}{c}{\textbf{Tiempo \small{(segundos)}}} \\
 \hline \hline
 5 & 12,44\\
 \hline
 10 & 15,03\\
 \hline
 \textbf{15} & \textbf{18,15}\\
 \hline
 20 & 28,45\\
 \hline
 25 & 43,56\\
 \hline
 \textbf{30} & \textbf{53,22}\\
 \hline \hline
\end{tabular}
\caption{Tiempo de procesamiento en funci�n de la longitud de las oraciones}
\label{tab:longitud-oraciones}
\end{center}
\end{table}

Como se puede apreciar en el caso resaltado, duplicar el tama�o medio de
las oraciones casi triplica el tiempo de procesamiento.

\section{Otros aspectos a tener en cuenta}

Con vistas a realizar un an�lisis m�s exhaustivo hay otras cuestiones que ser�a
recomendable tener en cuenta. En primer lugar el estado de la base de datos es
interesante, habr�a que estudiar el coste de las
transacciones a realizar ya que no es siempre uniforme\footnote{Por ejemplo, no
es lo mismo buscar una entrada en una tabla con un registro que con un mill�n}.
Seguidamente, tambi�n ser�a interesante buscar una m�quina con carga media lo
m�s cercana a cero, para logar un mayor rigor en las pruebas. Finalmente,
tambi�n ayudar�a disgregar en la medida de lo posible los tiempos totales en
funci�n de cada tarea que realiza el programa (filtros, an�lisis, etc�tera).




