\section{Herramientas utilizadas}

\subsection{Bibliotecas}

\begin{itemize}
\item \textbf{Stanford
Parser}\footnote{http://nlp.stanford.edu/software/lex-parser.shtml}.
Completo analizador de lenguaje natural desarrollado en la Universidad de
Stanford.
\item \textbf{JSAP}\footnote{http://www.martiansoftware.com/jsap/}. Sencillo
analizador de opciones introducidas a trav�s de la l�nea de comandos.
\item \textbf{MySQL
Connector/J}\footnote{http://www.mysql.com/products/connector/j/}. Conjunto de
clases para poder operar con MySQL utilizando Java.
\item \textbf{SAX}\footnote{http://www.saxproject.org/}. Un poderoso analizador
de ficheros XML, utilizando para leer y tratar los ficheros de entrada.
\end{itemize}

Todas estas librer�as de terceros son de libre distribuci�n.

\subsection{Herramientas de construcci�n}

\begin{itemize}
\item \textbf{Ant}\footnote{http://ant.apache.org/}. La utilidad m�s
utilizada para crear guiones de construcci�n en XML para proyectos de software.
Especialmente �til para proyectos hechos en Java.
\item \textbf{GNU Make}\footnote{http://www.gnu.org/software/make/}.
Alternativamenta tambi�n se proporciona una v�a para construir el proyecto
utilizando un Makefile.
\end{itemize}

Estas herramientas tambi�n son de libre distribuci�n.
