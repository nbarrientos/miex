\section{Herramientas utilizadas}

\subsection{Bibliotecas}

\begin{itemize}
\item \textbf{Stanford
Parser}\footnote{http://nlp.stanford.edu/software/lex-parser.shtml}.
Completo analizador de lenguaje natural escrito en Java y  desarrollado en la
Universidad de Stanford. (GPL)
\item \textbf{JSAP}\footnote{http://www.martiansoftware.com/jsap/}. Sencillo
analizador de opciones introducidas a trav�s de la l�nea de comandos. (LGPL -
con excepciones)
\item \textbf{MySQL
Connector/J}\footnote{http://www.mysql.com/products/connector/j/}. Conjunto de
clases para poder operar con MySQL utilizando Java. (GPL)
\item \textbf{SAX}\footnote{http://www.saxproject.org/}. Un poderoso analizador
de ficheros XML, utilizado para leer y tratar los ficheros de entrada.
(Dominio p�blico)
\end{itemize}

\subsection{Herramientas de construcci�n}

\begin{itemize}
\item \textbf{Ant}\footnote{http://ant.apache.org/}. Apache Ant es la
herramienta m�s utilizada para crear guiones de construcci�n en XML para
proyectos de software. Especialmente �til para proyectos hechos en Java.
(Apache)
\item \textbf{GNU Make}\footnote{http://www.gnu.org/software/make/}.
Alternativamente tambi�n se proporciona una v�a para construir el proyecto
utilizando un Makefile. (GPL)
\end{itemize}

