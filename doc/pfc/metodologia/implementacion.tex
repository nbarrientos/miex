\section{Implementaci�n}

\subsection{Diagrama de clases}

\subsection{M�dulos}

\subsubsection{Config}
\subsubsection{Datastr}
\subsubsection{Filter}
\subsubsection{Input}
\subsubsection{Run}
\subsubsection{Semantic}
\subsubsection{Sql}
\subsubsection{Tools}

\subsection{API}

Como puede observar, esta secci�n no trata en profuncidad cada elemento del
c�digo, es decir, no se describen profundamente cada una de las clases, m�todos,
etc�tera, m�s bien se hace una descripci�n abstrata de cada m�dulo del programa.
La raz�n de esto, es que junto a MIEX se distribuye documentaci�n del c�digo
generada con Javadoc\footnote{http://java.sun.com/j2se/javadoc/} que puede ser
utilizada como referencia para posibles futuros usos del proyecto.

Para generar la documentaci�n, no tiene nada m�s que ejecutar \texttt{make doc}
(si usa GNU Make) o \texttt{ant doc} (si usa Ant) en el diretorio ra�z donde
tenga las fuentes del programa.

Una vez construida, la documentaci�n se encuenta en el directorio
\texttt{build/doc/api} y puede ser consultada con cualquier int�rprete de HTML,
por ejemplo, un navegador web (Mozilla Firefox, Konqueror, Safari, Links...).

