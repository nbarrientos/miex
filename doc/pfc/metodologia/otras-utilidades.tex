\section{Otras utilidades}

\subsection{Scripts de transformaci�n}

Aunque estos scripts tienen una funcionalidad muy concreta y han sido dise�ados
para un caso de uso muy especial se distribuyen igualmente por si resultaran
necesarios. \textbf{No los utilice si no sabe bien lo que est� haciendo.}

Durante el periodo de pruebas se detect� la existencia de algunos ficheros de
entrada que no estaban bien formados, por lo que el analizador no los era capaz
de tratar correctamente. El script llamado \textbf{\texttt{fixxml.sh}} situado
en el directorio \texttt{share/utils} toma como argumento un fichero a arreglar
e imprime el fichero correcto por la salida est�ndar.

Los errores que �ste script arregla son los siguientes:

\begin{itemize}
	\item A�ade un nodo ra�z al ficheros de
colecci�n (\texttt{\textless{}collection\textgreater{}} y
\texttt{\textless{}/collection\textgreater{}})
	\item Finaliza correctamente las etiquetas de las secciones
\texttt{doctrain} y \texttt{doctest} de ficheros de colecci�n err�neos
(por ejemplo, los que comiencen por
\texttt{\textless{}doctrain\textgreater{}} y acaben con
\texttt{\textless{}/doc\textgreater{}}, se cambiar�a esta �ltima etiqueta por
\texttt{\textless{}/doctrain\textgreater{}})
\end{itemize}

Recuerde, MIEX \textbf{abandonar� la ejecuci�n} si alguno de los ficheros de
entrada no cumple el \textit{schema} proporcionado. Si usted fuerza evitar la
comprobaci�n utilizando \textit{schema} y el fichero de colecci�n est� mal
formado, el \textit{unmarshaller} lanzar� una excepci�n no recuperable y
abortar� la ejecuci�n.

Puede obtener m�s informaci�n acerca del formato correcto de los ficheros de
entrada en la secci�n \ref{sec:schema} de este mismo libro.

\subsection{Integraci�n con FK}

Como herramienta experimental, en el directorio \texttt{misc/fk} se distribuye
un \textit{wrapper} para poder invocar f�cilmente MIEX (y por consiguiente la
JVM\footnote{Java Virtual Machine})
desde FK\footnote{http://www.aic.uniovi.es/pir/Software.htm}.

Hasta el momento no existe ninguna manera de configurar el \textit{wrapper} en
tiempo de ejecuci�n por lo que probablemente tendr� que modificar alg�n
\textit{path} en el fichero fuente antes de compilar y ejecutar para que
funcione correctamente.

Finalmente, el autor de esta documentaci�n \textbf{no recomienda} lanzar MIEX
utilizando FK, utilice el sistema habitual para realizar las ejecuciones.