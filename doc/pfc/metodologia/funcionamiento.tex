\section{Funcionamiento interno}

Como puede observar en la figura \ref{fig:esquema-funcionamiento}, lo que
hasta ahora se ha comentado en esta secci�n no trata
en profundidad cada elemento del c�digo fuente, es decir, no se describen
minuciosamente cada una de las clases, m�todos, etc�tera, m�s bien se hace una
descripci�n abstrata de cada m�dulo del programa. La raz�n de esto, es que junto
a MIEX se distribuye documentaci�n del c�digo generada con
Javadoc\footnote{http://java.sun.com/j2se/javadoc/} que puede ser utilizada como
referencia para posibles usos posteriores del proyecto.

Para generar la documentaci�n, no tiene nada m�s que ejecutar \texttt{make doc}
(si usa GNU Make) o \texttt{ant doc} (si usa Ant) en el diretorio ra�z donde
tenga las fuentes del programa.

\begin{figure}[p]
\label{fig:esquema-funcionamiento}
\begin{center}
\includegraphics[width=347pt,height=690pt]{images/build/evolution.pdf}
% evolution.png: 694x1380 pixel, 51dpi, 34.70x69.00 cm, bb=0 0 984 1956
\end{center}
\caption{Esquema de funcionamiento interno}
\end{figure} 