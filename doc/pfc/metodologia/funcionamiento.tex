\section{Funcionamiento interno}

Como puede observar en la figura \ref{fig:esquema-funcionamiento}, en primer
lugar se analiza el contenido tanto de las opciones pasadas a trav�s de la
l�nea de comandos, como a trav�s del fichero de configuraci�n. Seguidamente, se
comprueba la validez de los ficheros que el usuario demanda procesar.
Obligatoriamente se eliminan ficheros que no existen y, si el usuario as� lo
solicita, los ficheros que no cumplan el \textit{schema} seleccionado. Para
finalizar este primer bloque de acciones, si despu�s de los filtros anteriores,
existen ficheros que procesar se crea una nueva base de datos, si as� se
hubiese especificado en el fichero de configuraci�n.

Si todo ha ido bien, se procede con el an�lisis de las colecciones siguiendo el
cauce que se establece en la figura.

\begin{figure}[p]
\label{fig:esquema-funcionamiento}
\begin{center}
%\includegraphics[width=347pt,height=690pt]{images/build/evolution.pdf}
\includegraphics[scale=0.35]{images/build/evolution.pdf}
% evolution.png: 694x1380 pixel, 51dpi, 34.70x69.00 cm, bb=0 0 984 1956
\end{center}
\caption{Esquema de funcionamiento interno}
\end{figure}
 
Como paso opcional dentro del proceso de tratamiento, existe la posibilidad de
crear un fichero de texto por cada categor�a encontrada con todas las oraciones
que se vayan analizando pertenecientes a esa categor�a. El proceso de filtrado
trata de eliminar palabras vac�as (m�s usualmente conocidas como \textit{stop
words}\footnote{http://en.wikipedia.org/wiki/Stop\_words}) y eliminar elementos
in�tiles del an�lisis, por ejemplo, signos de puntuaci�n.

Como se observa en los pasos finales, tambi�n es opcional la normalizaci�n de
las palabras analizadas. B�sicamente el proceso de normalizaci�n se trata de
aplicar el conocido algoritmo de Porter (en ingl�s, \textit{Porter
Stemmer algorithm}\footnote{http://en.wikipedia.org/wiki/Stemming}). De cara a
analizar los resultados posteriormente, una palabra normalizada de una que no lo
est� se distingue con una marca en la base de datos, por lo que realizar filtros
resulta bastante sencillo.

\newpage