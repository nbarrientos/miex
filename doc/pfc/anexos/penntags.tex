\chapter{Penn Treebank II Tags\label{sec:penntags}}

En el siguiente anexo se explica en ingl�s el significado de cada una de las
etiquetas.

\subsection*{Bracket Labels}

\subsubsection*{Clause Level}

\par 

\textbf{S} - simple declarative clause, i.e. one that is not introduced by a
(possible empty) subordinating conjunction or a \textit{wh}-word and that does
not exhibit subject-verb inversion.\\

\textbf{SBAR} - Clause introduced by a (possibly empty) subordinating
conjunction.\\

\textbf{SBARQ} - Direct question introduced by a \textit{wh}-word or
a \textit{wh}-phrase.  Indirect questions and relative clauses should be
bracketed as SBAR, not SBARQ.\\



\textbf{SINV} - Inverted declarative sentence, i.e. one in which the subject follows the tensed verb or modal.\\



\textbf{SQ} - Inverted yes/no question, or main clause of a \textit{wh}-question, following the \textit{wh}-phrase in SBARQ.\\



\subsubsection*{Phrase Level}

\textbf{ADJP} - Adjective Phrase.\\



\textbf{ADVP} - Adverb Phrase.\\



\textbf{CONJP} - Conjunction Phrase.\\



\textbf{FRAG} - Fragment.\\



\textbf{INTJ} - Interjection.  Corresponds approximately to the part-of-speech tag UH.\\



\textbf{LST} - List marker.  Includes surrounding punctuation.\\



\textbf{NAC} - Not a Constituent; used to show the scope of certain prenominal modifiers within an NP.\\



\textbf{NP} - Noun Phrase. \\



\textbf{NX} - Used within certain complex NPs to mark the head of the NP.  Corresponds very roughly to N-bar level but used  
quite differently.\\



\textbf{PP} - Prepositional Phrase.\\



\textbf{PRN} - Parenthetical. \\



\textbf{PRT} - Particle.  Category for words that should be tagged RP. \\



\textbf{QP} - Quantifier Phrase (i.e. complex measure/amount phrase); used within NP.\\



\textbf{RRC} - Reduced Relative Clause. \\



\textbf{UCP} - Unlike Coordinated Phrase. \\



\textbf{VP} - Vereb Phrase. \\



\textbf{WHADJP} - \textit{Wh}-adjective Phrase.  Adjectival phrase containing a \textit{wh}-adverb, as in \textit{how hot}.\\



\textbf{WHAVP} - \textit{Wh}-adverb Phrase.  Introduces a clause with an NP gap.  May be null (containing the 0 complementizer)  
or lexical, containing a \textit{wh}-adverb such as \textit{how} or \textit{why}.\\



\textbf{WHNP} - \textit{Wh}-noun Phrase.  Introduces a clause with an NP gap.  May be null (containing the 0 complementizer) or  
lexical, containing some \textit{wh}-word, e.g. \textit{who}, \textit{which book}, \textit{whose daughter}, \textit{none of which},  
or \textit{how many leopards}.\\



\textbf{WHPP} - \textit{Wh}-prepositional Phrase.  Prepositional phrase containing a \textit{wh}-noun phrase (such as  
\textit{of which} or \textit{by whose authority}) that either introduces a PP gap or is contained by a WHNP.\\



\textbf{X} - Unknown, uncertain, or unbracketable.  X is often used for bracketing typos and in bracketing  
\textit{the...the}-constructions.\\



\subsubsection*{Word level}


\textbf{CC}   - Coordinating conjunction\\


\textbf{CD}   - Cardinal number\\


\textbf{DT}   - Determiner\\


\textbf{EX}   - Existential there\\


\textbf{FW}   - Foreign word\\


\textbf{IN}   - Preposition or subordinating conjunction\\


\textbf{JJ}   - Adjective\\


\textbf{JJR}  - Adjective, comparative\\


\textbf{JJS}  - Adjective, superlative\\


\textbf{LS}   - List item marker\\


\textbf{MD}   - Modal\\


\textbf{NN}   - Noun, singular or mass\\


\textbf{NNS}  - Noun, plural\\


\textbf{NNP}  - Proper noun, singular\\


\textbf{NNPS} - Proper noun, plural\\


\textbf{PDT}  - Predeterminer\\


\textbf{POS}  - Possessive ending\\


\textbf{PRP}  - Personal pronoun\\


\textbf{PRP\$} - Possessive pronoun (prolog version PRP-S)\\


\textbf{RB}   - Adverb\\


\textbf{RBR}  - Adverb, comparative\\


\textbf{RBS}  - Adverb, superlative\\


\textbf{RP}   - Particle\\


\textbf{SYM}  - Symbol\\


\textbf{TO}   - to\\


\textbf{UH}   - Interjection\\


\textbf{VB}   - Verb, base form\\


\textbf{VBD}  - Verb, past tense\\


\textbf{VBG}  - Verb, gerund or present participle\\


\textbf{VBN}  - Verb, past participle\\


\textbf{VBP}  - Verb, non-3rd person singular present\\


\textbf{VBZ}  - Verb, 3rd person singular present\\


\textbf{WDT}  - Wh-determiner\\


\textbf{WP}   - Wh-pronoun\\


\textbf{WP\$}  - Possessive wh-pronoun (prolog version WP-S)\\


\textbf{WRB}  - Wh-adverb\\
