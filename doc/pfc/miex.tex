\documentclass[spanish,a4paper,12pt,twosides]{book}

%\usepackage[utf8]{inputenc}
%\usepackage[T1]{fontenc}
\usepackage[latin1]{inputenc}
\usepackage[margin=1in]{geometry}
%\usepackage{paralist}
\usepackage{graphicx}
\usepackage{url}
\usepackage{times}
\usepackage{babel}
\usepackage{listings}
\usepackage{verbatim}
\usepackage{float}
\usepackage{listings}
\usepackage{color}
\usepackage{array}
%\usepackage{eurosym}
\usepackage[strict]{chngpage}

\pagestyle{headings}

% Some subsvars
\newcommand{\TituloPFC}{Extracci�n de informaci�n de documentos XML mediante
Stanford Parser}
\newcommand{\TituloPFCWeb}{MIEX: Metadata and Information Extractor from small
XML documents}
\newcommand{\AutorPFC}{Ignacio Barrientos Arias}
\newcommand{\Escuela}{Escuela Universitaria de Ingenier�a T�cnica Inform�tica
de Gij�n}
\newcommand{\Universidad}{Universidad de Oviedo}
\newcommand{\Convocatoria}{SEPTIEMBRE 2007}

\newcommand{\sourcepath}
{/home/nacho/devel/pfc/svn/trunk/miex/src/es/uniovi/aic/miex/}
\newcommand{\trunkpath}
{/home/nacho/devel/pfc/svn/trunk/miex/}
\newcommand{\codeversion}
{0.1.0}

% Some colour definitions
\definecolor{listinggray}{gray}{0.97}

% Cabeceras
\headheight 15pt

% Listing settings
\lstset
{
	basewidth=0.50em,
	backgroundcolor=\color{listinggray},
	basicstyle=\footnotesize\ttfamily,
	keywordstyle=\bfseries,
	stringstyle=\itshape,
	commentstyle=\itshape,
	showspaces=false,
	showtabs=false,
	showstringspaces=false,
	frame=trbl,
	extendedchars=true,
	%aboveskip=0.5cm,
	%belowskip=0.5cm,
	xleftmargin=0cm,
	xrightmargin=0cm,
	tabsize=1,
	numbers=left, 
	numberstyle=\small,
	stepnumber=9,
	numbersep=5pt
}

\parskip  \medskipamount

% Titlesec
\titleformat{\chapter}[frame]
  {\fontfamily{phv}}
  {\filright\large\enspace \chaptertitlename \enspace \thechapter\enspace}
  {8pt}
  {\huge\bfseries\filcenter}
\titleformat{\section}
 {\normalfont\selectfont\large}
 {\thesection}{1em}{\bfseries}[\titlerule]
\titleformat{\section}
  {\fontfamily{phv}\selectfont\Large}
  {\thesection}{1em}{\bfseries}[\titlerule]
\titleformat{\subsection}
  {\fontfamily{phv}\selectfont\large}
  {\thesubsection}{1em}{\bfseries}

%% redefinir la construccion del titulo para hacer una portada acorde a las
% espeficiaciones de la escuela

\renewcommand{\maketitle}
{
    \begin{titlepage}

	%portada del tomo
	\begin{center}
                {%
                    \Large\textrm{\textbf{UNIVERSIDAD DE OVIEDO}}\\
                }
                {%
                    \Large\textrm{Escuela Universitaria de Ingenier�a T�cnica Inform�tica de Gij�n}\\
                    \vspace{4cm}
                }
                {%
                    \Large\textrm{\textbf{PROYECTO FIN DE CARRERA}}\\
                    \vspace{5cm}
                }
                {%
                    \Large\textrm{``\TituloPFC''}\\
                    \vspace{5cm}
                }
	\end{center}
	\begin{flushright}
		\Large\textrm{\AutorPFC}\\
		\Large\textrm{Junio 2007}
	\end{flushright}
	\pagestyle{empty}
	
	\cleardoublepage

	%portada de dentro
	\begin{center}
                {%
                    \Large\textrm{\textbf{UNIVERSIDAD DE OVIEDO}}\\
                    \vspace{2cm}
                }
                {%
                    \includegraphics[width=3.5cm]{images/uniovi.png} \\
                }
                {%
                    \Large\textrm{ESCUELA UNIVERSITARIA DE INGENIER�A T�CNICA EN INFORM�TICA DE GIJ�N}\\
                    \vspace{3cm}
                }
                {%
                    \Large\textrm{\textbf{PROYECTO FIN DE CARRERA}}\\
                    \vspace{3cm}
                }
                {%
                    \Large\textrm{\uppercase{''\TituloPFC``}}\\
                    \vspace{3cm}
                }
		{
		    \Large\textrm{DIRECTORAS}\\
		    \Large\textrm{\DirectorPFC}\\
		    \Large\textrm{\CoDirectorPFC}\\
		    \vspace{2cm}
		}
		{
		    \Large\textrm{AUTOR \\ \AutorPFC}\\
		}
	
	\end{center}
	\begin{flushright}
		\Large\textrm{Junio 2007}
	\end{flushright}
        \pagestyle{empty}
	
	\cleardoublepage
    
    \end{titlepage}
}


\title{\TituloPFCnorm}
\author{\AutorPFC}
\date{Junio de 2007}

\begin{document}

\frontmatter

\maketitle

%\bibliographystyle{plain}

\chapter*{Resumen}

Documentacion del proyecto titulado \emph{\TituloPFCnorm}, 

\section*{Palabras clave}

Recuperaci�n de informaci�n, sem�ntica, Java, analizador sint�ctico

\chapter*{Agradecimientos}

A todas las mujere que han hecho posible la realizaci�n de este
proyecto XDD

\chapter*{Licencia}

\section*{Documento y c�digo fuente}

El c�digo fuente (disponible en el anexo \ref{sec:source}) se encuentra 
licenciado bajo la licencia \emph{GNU General Public License (GPL)}, 
versi�n 2 (anexo \ref{sec:license.gpl}).

\chapter*{Historial de este documento}

\begin{tabular}{|l|l|l|}
 \hline
 \textbf{Fecha} & \textbf{Versi�n} & \textbf{Comentarios} \\\hline
 Apr/2007 & 0.1 & Primer borrador \\\hline
\end{tabular}

\tableofcontents

\newpage

\listoffigures

\newpage

\listoftables

\newpage

\mainmatter

\chapter{Memoria}

\section{T�tulo del proyecto}

Si bien el t�tulo original del proyecto es \textit{MIEX: Metadata and
Information Extractor from small XML documents}, para la distribuci�n en el
marco universitario se ha optado por el siguiente t�tulo:

\emph{\TituloPFC}.

\section{Alcance de este documento}

Este libro pretende explicar lo m�s breve y concisamente posible como se ha
desarrollado MIEX as� como dar las claves necesarias para utilizarlo.

Adem�s de una peque�a memoria que abarcar� los aspectos no t�cnicos y el
historial del proceso de desarrollo, se incluye un cap�tulo completo destinado
a personal t�cnico (principalmente desarrolladores) y un manual para usuarios
finales.


\chapter{Introducci�n}

\section{T�tulo del proyecto}

Si bien el t�tulo original del proyecto es \textit{MIEX: Metadata and
Information Extractor from small XML documents}, para la distribuci�n en el
marco universitario se ha optado por el siguiente t�tulo:

\emph{\TituloPFC}.

\section{Alcance de este documento}

Este libro pretende explicar lo m�s breve y concisamente posible como se ha
desarrollado MIEX as� como dar las claves necesarias para utilizarlo.

Adem�s de una peque�a memoria que abarcar� los aspectos no t�cnicos y el
historial del proceso de desarrollo, se incluye un cap�tulo completo destinado
a personal t�cnico (principalmente desarrolladores) y un manual para usuarios
finales.


\section{Primeros pasos}

Uno de los campos en los cuales se hace m�s hincapi� dentro de la recuperaci�n
de informaci�n es el an�lisis, tanto sint�ctico como sem�ntico, de textos para
posteriormente poder analizar los resultados obtenidos y, por ejemplo, poder
buscar palabras que representen a una serie de documentos.

La construcci�n de analizadores de lenguaje natural es una tarea, que, adem�s
de ser compleja, el propio an�lisis tiene un conste computacional muy elevado
que fuerza al investigador a tener que utilizar equipos de alto rendimiento
para obtener resultados en un tiempo razonable.

El objetivo de este proyecto es la elaboraci�n de un envoltorio
(\textit{wrapper}) de uno de los analizadores m�s conocidos, el Stanford
Parser. Con �ste envoltorio se pretende hacer que el manejo del mismo sea lo
m�s sencillo posible, para que, la persona que lo use, se pueda abstraer
totalmente del funcionamiento interno del software y poder analizar documentos
f�cilmente.

\begin{figure}[h]
\begin{center}
	\includegraphics[bb=0 0 148 240]{images/miex.png}
	% miex.png: 257x416 pixel, 125dpi, 5.22x8.45 cm, bb=0 0 148 240
\end{center}
\caption{Esquema de funcionamiento de MIEX}
\end{figure} 





\section{Objetivos}

Los objetivos de este proyecto son los siguientes:

\begin{itemize}
	\item An�lisis de documentos XML.
	\item Comprobaci�n de la integridad de los documentos XML.
	\item Utilizaci�n de ficheros de configuraci�n.
	\item Uso del Stanford Parser para extraer toda la informaci�n posible
de los documentos analizados.
	\item An�lisis de la informaci�n recabada por el Stanford Parser.
	\item Limpieza de informaci�n in�til.
	\item Dise�o de una base de datos para guardar la informaci�n extraida.
\end{itemize}


\section{Entradas}

En la siguiente secci�n se explicar�n todos los aspectos relevantes a los
ficheros de entrada que procesar� MIEX.

\subsection{XML}

Los ficheros de entrada est� en formato XML, en primer lugar vamos a ver qu� es
XML. La definici�n, extraida
de Wikipedia\footnote{http://es.wikipedia.org/wiki/XML} es la siguiente:

\textit{�XML, sigla en ingl�s de eXtensible Markup Language (�lenguaje de
marcas extensible�), es un metalenguaje extensible de etiquetas desarrollado
por el World Wide Web Consortium (W3C). Es una simplificaci�n y adaptaci�n
del SGML y permite definir la gram�tica de lenguajes espec�ficos (de la
misma manera que HTML es a su vez un lenguaje definido por SGML). Por lo
tanto XML no es realmente un lenguaje en particular, sino una manera de
definir lenguajes para diferentes necesidades�.}

\subsection{El lenguaje utilizado}

El lenguaje utilizado en los ficheros de entrada corresponde al siguiente
\textit{schema}

\begin{Verbatim}[fontsize=\relsize{-2}]
<?xml version="1.0"?>
<xs:schema xmlns:xs="http://www.w3.org/2001/XMLSchema">

<xs:complexType name="categoriestype">
 <xs:sequence>

  <xs:element name="D" type="xs:string" minOccurs="1" maxOccurs="unbounded"/>

 </xs:sequence>
</xs:complexType>

<xs:complexType name="doctype">
  <xs:all>

  <xs:element name="TOPIC" type="categoriestype" minOccurs="1" maxOccurs="1"/>

  <xs:element name="TITLE" type="xs:string" maxOccurs="1"/>

  <xs:element name="BODY" type="xs:string" minOccurs="1" maxOccurs="1"/>

  </xs:all>
</xs:complexType>

<xs:element name="COLLECTION">
<xs:complexType>
  <xs:sequence>

    <xs:element name="DOCTRAIN" type="doctype" minOccurs="0"
maxOccurs="unbounded"/>

    <xs:element name="DOCTEST" type="doctype" minOccurs="0"
maxOccurs="unbounded"/>

  </xs:sequence>
</xs:complexType>
</xs:element>

</xs:schema>
\end{Verbatim}

Por lo que cada fichero que represente una colecci�n de documentos ser�
gen�ricamente de esta forma:

\begin{Verbatim}[fontsize=\relsize{-2}]
<collection>
 <doctrain>
 <topic> <d>c1</d> <d>c2</d> ... <d>cn</d> </topic>
 <title>un t�tulo</title>
 <body>un cuerpo</body>
 </doctrain>
 ...
 <doctrain>
 <topic> <d>c1</d> <d>c2</d> ... <d>cn</d> </topic>
 <title>otro t�tulo</title>
 <body>otro cuerpo</body>
 </doctrain>
</collection>
 \end{Verbatim}
 
\subsection{�Qu� informaci�n es importante?}

De cara a analizar la sem�tica del contenido es importante la informaci�n
comprendida en las etiquetas \textit{title} y \textit{body}. Por otro lado, las
categor�as de cada documento tambi�n tienen importancia, ya que en un futuro
podr�a ser interesante analizar la informaci�n obtenida en funci�n a la
categor�a, por ejemplo, para hacer agrupamientos.

\subsection{Ejemplo de una colecci�n}

A continuaci�n, un peque�o ejemplo de uno fichero de colecci�n.

\begin{Verbatim}[fontsize=\relsize{-2}]
<collection>
 <doctrain>
 <topic> <d>debian</d> <d>linux</d> <d>OS</d> </topic>
 <title>Debian project</title>
 <body> Debian is a free operating system (OS) for your computer. An operating
system is the set of basic programs and utilities that make your computer run.
Debian uses the Linux kernel (the core of an operating system), but most of the
basic OS tools come from the GNU project; hence the name GNU/Linux </body>
 </doctrain>
 <doctrain>
 <topic> <d>computing</d> <d>search</d> <d>engine</d> </topic>
 <title>Google</title>
 <body> Google Inc. (NASDAQ: GOOG and LSE: GGEA) is an American public
corporation, specializing in Internet search and online advertising. The company
had 10,674 full-time employees as of December 31, 2006, and is based in Mountain
View, California. </body>
 </doctrain>
 </collection>
\end{Verbatim}

Se puede apreciar que se trata de una colecci�n formada por dos documentos
titulados \textit{Debian project} y \textit{Google}. El primer documento se
encuentra en las categor�as \textit{debian}, \textit{linux} y \textit{OS} as�
como el segundo est� en las categor�as \textit{computing}, \textit{search}
y \textit{engine}.

\subsection{Adaptando ficheros mal formados}

Los ficheros de colecci�n para los que inicialmente fue dise�ado MIEX no ten�an
un formato correcto. Concretamente el XML no inclu�a nodo ra�z ni cerraba
correctamente las etiquetas. Para paliar este problema, se han desarrollado un
par de herramientas que permiten f�cilmente la adaptaci�n al formato correcto
de los ficheros problem�ticos. Se proporcionar� m�s informaci�n en los pr�ximos
cap�tulos.


\section{Almacenamiento de la informaci�n}

A la hora de decidir el medio para almacenar los resultados obtenidos
r�pidamente se descart� la opci�n de utilizar ficheros de texto plano, ya que
requerir�an la construcci�n de analizadores adiccionales para, una vez se
quiera consultar los resultados, poder recuperar la informaci�n.

Finalmente se decidi� utilizar una base de datos relacional para poder, en
cierta manera, dise�ar una estructura lo m�s c�moda posible para dejar abierto
un gran abanico de posibilidades a la hora de consultar la informaci�n obtenida.

En el cap�tulo que se dedica a los aspectos t�cnicos, se explica la base de
datos utilizada as� como la estructura de la misma. 

\section{Recursos sobre el proyecto}

\subsection{P�gina web}

Una vez sea presentado el proyecto la intenci�n del autor es la de seguir
desarrollando MIEX en sus ratos libres. Para hacer el trabajo m�s transparente,
existe un proyecto creado en SourceForge\footnote{http://www.sourceforge.net}
as� como una p�gina web en la siguiente direcci�n.

\url{http://miex.sf.net}

A trav�s de ese URI\footnote{Uniform Resource Identifier} es posible obtener la
informaci�n m�s actualizada acerca de MIEX, estar al d�a de los eventos que
rodean al proyecto y obtener las �ltimas \textit{snapshots} del c�digo fuente.

\subsection{RSS}

Si desea suscribirse al canal RSS\footnote{http://es.wikipedia.org/wiki/RSS} que
distribuye las noticias sobre el proyecto no tiene m�s que a�adir la siguiente
direcci�n en su agregador favorito:

\url{http://sourceforge.net/export/rss2_projnews.php?group_id=187602}

\subsection{Subversion}

El repositorio donde se lleva el control de las versiones del c�digo fuente
(utilizando Subversion\footnote{http://subversion.tigris.org/}) se
encuentra en \url{https://miex.svn.sourceforge.net/svnroot/miex} y tiene la
siguiente estructura:

\begin{itemize}
	\item \textbf{branches} - Ramas de desarrollo paralelas a
\textit{trunk}
	\item \textbf{tags} - Versiones congeladas (releases)
	\item \textbf{trunk} - Rama principal de desarrollo de MIEX 
	\begin{itemize}
		\item \textbf{doc} - Documentaci�n (presentaciones, libros...)
		\item \textbf{miex} - C�digo fuente del proyecto
		\item \textbf{web} - P�gina web
	\end{itemize}
\end{itemize}

\textbf{Nota:} Use el contenido de \textit{trunk} con precauci�n. Debido a que
es c�digo que suele cambiar con mucha frecuencia, puede descargar una captura
inestable y obtener resultados inesperados. Si desea una versi�n estable y
funcional desc�rgela directamente de la p�gina web o obt�ngala desde
\textit{tags}.



\newpage




\section{Objetivos}

Los objetivos de este proyecto son los siguientes:

\begin{itemize}
	\item An�lisis de documentos XML.
	\item Comprobaci�n de la integridad de los documentos XML.
	\item Utilizaci�n de ficheros de configuraci�n.
	\item Uso del Stanford Parser para extraer toda la informaci�n posible
de los documentos analizados.
	\item An�lisis de la informaci�n recabada por el Stanford Parser.
	\item Limpieza de informaci�n in�til.
	\item Dise�o de una base de datos para guardar la informaci�n extraida.
\end{itemize}


\section{Entradas}

En la siguiente secci�n se explicar�n todos los aspectos relevantes a los
ficheros de entrada que procesar� MIEX.

\subsection{XML}

Los ficheros de entrada est� en formato XML, en primer lugar vamos a ver qu� es
XML. La definici�n, extraida
de Wikipedia\footnote{http://es.wikipedia.org/wiki/XML} es la siguiente:

\textit{�XML, sigla en ingl�s de eXtensible Markup Language (�lenguaje de
marcas extensible�), es un metalenguaje extensible de etiquetas desarrollado
por el World Wide Web Consortium (W3C). Es una simplificaci�n y adaptaci�n
del SGML y permite definir la gram�tica de lenguajes espec�ficos (de la
misma manera que HTML es a su vez un lenguaje definido por SGML). Por lo
tanto XML no es realmente un lenguaje en particular, sino una manera de
definir lenguajes para diferentes necesidades�.}

\subsection{El lenguaje utilizado}

El lenguaje utilizado en los ficheros de entrada corresponde al siguiente
\textit{schema}

\begin{Verbatim}[fontsize=\relsize{-2}]
<?xml version="1.0"?>
<xs:schema xmlns:xs="http://www.w3.org/2001/XMLSchema">

<xs:complexType name="categoriestype">
 <xs:sequence>

  <xs:element name="D" type="xs:string" minOccurs="1" maxOccurs="unbounded"/>

 </xs:sequence>
</xs:complexType>

<xs:complexType name="doctype">
  <xs:all>

  <xs:element name="TOPIC" type="categoriestype" minOccurs="1" maxOccurs="1"/>

  <xs:element name="TITLE" type="xs:string" maxOccurs="1"/>

  <xs:element name="BODY" type="xs:string" minOccurs="1" maxOccurs="1"/>

  </xs:all>
</xs:complexType>

<xs:element name="COLLECTION">
<xs:complexType>
  <xs:sequence>

    <xs:element name="DOCTRAIN" type="doctype" minOccurs="0"
maxOccurs="unbounded"/>

    <xs:element name="DOCTEST" type="doctype" minOccurs="0"
maxOccurs="unbounded"/>

  </xs:sequence>
</xs:complexType>
</xs:element>

</xs:schema>
\end{Verbatim}

Por lo que cada fichero que represente una colecci�n de documentos ser�
gen�ricamente de esta forma:

\begin{Verbatim}[fontsize=\relsize{-2}]
<collection>
 <doctrain>
 <topic> <d>c1</d> <d>c2</d> ... <d>cn</d> </topic>
 <title>un t�tulo</title>
 <body>un cuerpo</body>
 </doctrain>
 ...
 <doctrain>
 <topic> <d>c1</d> <d>c2</d> ... <d>cn</d> </topic>
 <title>otro t�tulo</title>
 <body>otro cuerpo</body>
 </doctrain>
</collection>
 \end{Verbatim}
 
\subsection{�Qu� informaci�n es importante?}

De cara a analizar la sem�tica del contenido es importante la informaci�n
comprendida en las etiquetas \textit{title} y \textit{body}. Por otro lado, las
categor�as de cada documento tambi�n tienen importancia, ya que en un futuro
podr�a ser interesante analizar la informaci�n obtenida en funci�n a la
categor�a, por ejemplo, para hacer agrupamientos.

\subsection{Ejemplo de una colecci�n}

A continuaci�n, un peque�o ejemplo de uno fichero de colecci�n.

\begin{Verbatim}[fontsize=\relsize{-2}]
<collection>
 <doctrain>
 <topic> <d>debian</d> <d>linux</d> <d>OS</d> </topic>
 <title>Debian project</title>
 <body> Debian is a free operating system (OS) for your computer. An operating
system is the set of basic programs and utilities that make your computer run.
Debian uses the Linux kernel (the core of an operating system), but most of the
basic OS tools come from the GNU project; hence the name GNU/Linux </body>
 </doctrain>
 <doctrain>
 <topic> <d>computing</d> <d>search</d> <d>engine</d> </topic>
 <title>Google</title>
 <body> Google Inc. (NASDAQ: GOOG and LSE: GGEA) is an American public
corporation, specializing in Internet search and online advertising. The company
had 10,674 full-time employees as of December 31, 2006, and is based in Mountain
View, California. </body>
 </doctrain>
 </collection>
\end{Verbatim}

Se puede apreciar que se trata de una colecci�n formada por dos documentos
titulados \textit{Debian project} y \textit{Google}. El primer documento se
encuentra en las categor�as \textit{debian}, \textit{linux} y \textit{OS} as�
como el segundo est� en las categor�as \textit{computing}, \textit{search}
y \textit{engine}.

\subsection{Adaptando ficheros mal formados}

Los ficheros de colecci�n para los que inicialmente fue dise�ado MIEX no ten�an
un formato correcto. Concretamente el XML no inclu�a nodo ra�z ni cerraba
correctamente las etiquetas. Para paliar este problema, se han desarrollado un
par de herramientas que permiten f�cilmente la adaptaci�n al formato correcto
de los ficheros problem�ticos. Se proporcionar� m�s informaci�n en los pr�ximos
cap�tulos.


\newpage

%\input{memoria/web-semantica.tex}



\chapter{Metodolog�a y aspectos t�cnicos}

\section{Funcionamiento interno}

Como puede observar en la figura \ref{fig:esquema-funcionamiento}, en primer
lugar se analiza el contenido tanto de las opciones pasadas a trav�s de la
l�nea de comandos, como a trav�s del fichero de configuraci�n. Seguidamente, se
comprueba la validez de los ficheros que el usuario demanda procesar.
Obligatoriamente se eliminan ficheros que no existen y, si el usuario as� lo
solicita, los ficheros que no cumplan el \textit{schema} seleccionado. Para
finalizar este primer bloque de acciones, si despu�s de los filtros anteriores,
existen ficheros que procesar se crea una nueva base de datos, si as� se
hubiese especificado en el fichero de configuraci�n.

Si todo ha ido bien, se procede con el an�lisis de las colecciones siguiendo el
cauce que se establece en la figura.

\begin{figure}[p]
\label{fig:esquema-funcionamiento}
\begin{center}
%\includegraphics[width=347pt,height=690pt]{images/build/evolution.pdf}
\includegraphics[scale=0.35]{images/build/evolution.pdf}
% evolution.png: 694x1380 pixel, 51dpi, 34.70x69.00 cm, bb=0 0 984 1956
\end{center}
\caption{Esquema de funcionamiento interno}
\end{figure}
 
Como paso opcional dentro del proceso de tratamiento, existe la posibilidad de
crear un fichero de texto por cada categor�a encontrada con todas las oraciones
que se vayan analizando pertenecientes a esa categor�a. El proceso de filtrado
trata de eliminar palabras vac�as (m�s usualmente conocidas como \textit{stop
words}\footnote{http://en.wikipedia.org/wiki/Stop\_words}) y eliminar elementos
in�tiles del an�lisis, por ejemplo, signos de puntuaci�n.

Como se observa en los pasos finales, tambi�n es opcional la normalizaci�n de
las palabras analizadas. B�sicamente el proceso de normalizaci�n se trata de
aplicar el conocido algoritmo de Porter (en ingl�s, \textit{Porter
Stemmer algorithm}\footnote{http://en.wikipedia.org/wiki/Stemming}). De cara a
analizar los resultados posteriormente, una palabra normalizada de una que no lo
est� se distingue con una marca en la base de datos, por lo que realizar filtros
resulta bastante sencillo.

\newpage

\section{Herramientas utilizadas}

\subsection{Bibliotecas}

\begin{itemize}
\item \textbf{Stanford
Parser}\footnote{http://nlp.stanford.edu/software/lex-parser.shtml}.
Completo analizador de lenguaje natural escrito en Java y  desarrollado en la
Universidad de Stanford. (GPL)
\item \textbf{JSAP}\footnote{http://www.martiansoftware.com/jsap/}. Sencillo
analizador de opciones introducidas a trav�s de la l�nea de comandos. (LGPL -
con excepciones)
\item \textbf{MySQL
Connector/J}\footnote{http://www.mysql.com/products/connector/j/}. Conjunto de
clases para poder operar con MySQL utilizando Java. (GPL)
\item \textbf{SAX}\footnote{http://www.saxproject.org/}. Un poderoso analizador
de ficheros XML, utilizado para leer y tratar los ficheros de entrada.
(Dominio p�blico)
\end{itemize}

\subsection{Herramientas de construcci�n}

\begin{itemize}
\item \textbf{Ant}\footnote{http://ant.apache.org/}. Apache Ant es la
herramienta m�s utilizada para crear guiones de construcci�n en XML para
proyectos de software. Especialmente �til para proyectos hechos en Java.
(Apache)
\item \textbf{GNU Make}\footnote{http://www.gnu.org/software/make/}.
Alternativamente tambi�n se proporciona una v�a para construir el proyecto
utilizando un Makefile. (GPL)
\end{itemize}



\section{Implementaci�n}

\subsection{Diagrama de clases}

\subsection{M�dulos}

\subsubsection{Config}
\subsubsection{Datastr}
\subsubsection{Filter}
\subsubsection{Input}
\subsubsection{Run}
\subsubsection{Semantic}
\subsubsection{Sql}
\subsubsection{Tools}

\subsection{API}

Como puede observar, lo que hasta ahora se ha comentado en esta secci�n no trata
en profundidad cada elemento del c�digo fuente, es decir, no se describen
minuciosamente cada una de las clases, m�todos, etc�tera, m�s bien se hace una
descripci�n abstrata de cada m�dulo del programa. La raz�n de esto, es que junto
a MIEX se distribuye documentaci�n del c�digo generada con
Javadoc\footnote{http://java.sun.com/j2se/javadoc/} que puede ser utilizada como
referencia para posibles usos posteriores del proyecto.

Para generar la documentaci�n, no tiene nada m�s que ejecutar \texttt{make doc}
(si usa GNU Make) o \texttt{ant doc} (si usa Ant) en el diretorio ra�z donde
tenga las fuentes del programa.

Una vez construida, la documentaci�n se encuenta en el directorio
\texttt{build/doc/api} y puede ser consultada con cualquier int�rprete de HTML,
por ejemplo, un navegador web (Mozilla Firefox, Konqueror, Safari, Links...).



\section{Base de datos}

\subsection{Estructura}

Las entidades que se pueden localizar en la figura \ref{fig:diagrama-e-r} son
las siguientes.

\begin{itemize}
	\item Word
	\item Document
	\begin{itemize}
		\item DocType
	\end{itemize}
	\item Collection
	\item Property
	\begin{itemize}
		\item PropertyList
		\item PropType
	\end{itemize}
	\item Category
\end{itemize}

\begin{figure}[h]
\label{fig:diagrama-e-r}
\begin{center}
\includegraphics[angle=90,scale=0.425]{images/build/bdd-e-r.pdf}
\end{center}
\caption{Diagrama entidad-relaci�n de la base de datos}
\end{figure}

\subsection{Normalizaci�n}

\subsection{Vistas}



\section{Los ficheros de configuraci�n}

MIEX utiliza un fichero de configuraci�n de tipo \textit{clave=valor} para
personalizar el comportamiento del programa. El fichero de configuraci�n de
ejemplo que se distribuye con el software est� bastante explicado, por lo que en
esta secci�n simplemente se har� referencia a �l.

Las opciones de configuraci�n existentes hasta la fecha son las siguientes:

\begin{Verbatim}[fontsize=\relsize{-2}]
# BDHostname
#   Type:     String
#   Default:  -
#   Required: Yes
#   Desc:     Self-explaining.
#   See also: BD*

# BDUser
#   Type:     String
#   Default:  -
#   Required: Yes
#   Desc:     Self-explaining.
#   See also: BD*

# BDPassword
#   Type:     String
#   Default:  -
#   Required: Yes
#   Desc:     Self-explaining.
#   See also: BD*

# BDName
#   Type:     String
#   Default:  -
#   Required: Yes
#   Desc:     Self-explaining.
#   See also: BD*

# Validate
#   Type:     Yes/No (boolean)
#   Default:  No
#   Required: No
#   Desc:     Tries to validate each collection file using XML Schema.
#   See also: XMLschemaURI

# XMLschemaURI
#   Type:     String
#   Default:  /usr/share/miex/schemas/default.xsd
#   Required: No
#   Desc:     Path to a XML Schema used to validate XML inputs.
#   See also: Validate

# Dump
#   Type:     Yes/No (boolean)
#   Default:  No
#   Required: No
#   Desc:     Creates one file per category containing sentences.
#   See also: DumpDir

# DumpDir
#   Type:     String
#   Default:  /tmp
#   Required: No
#   Desc:     Path to a directory where the files will be dumped.
#   See also: Dump

# CreateDB
#   Type:     Yes/No (boolean)
#   Default:  No
#   Required: No
#   Desc:     Injects the template sql file into the database before do any
insert. USE CAREFULLY.
#   See also: SQLSkeleton

# SQLSkeleton
#   Type:     String
#   Default:  /usr/share/miex/sql/skeleton.sql
#   Required: No
#   Desc:     Path to an SQL file containing the DB creation statements
#   See also: CreateDB

# Normalize
#   Type:     Yes/No (boolean)
#   Default:  No
#   Required: No
#   Desc:     If 'Yes' MIEX uses Porter and inserts normalized words as well.
#   See also: -
\end{Verbatim}

Una vez instalado el programa como se indica en el siguiente cap�tulo, se puede
tomar como plantilla el fichero de ejemplo que se encuentra en el directorio:

\texttt{\$PREFIX/share/doc/miex/examples/conf}


\section{Otras utilidades}

\subsection{Scripts de transformaci�n}

Aunque estos scripts tienen una funcionalidad muy concreta y han sido dise�ados
para un caso de uso muy especial, se distribuyen igualmente por si resultaran
necesarios. \textbf{No los utilice si no sabe bien lo que est� haciendo.}

Durante el periodo de pruebas se detect� la existencia de algunos ficheros de
entrada que no estaban bien formados, por lo que el analizador no era capaz
de tratarlos correctamente. El script llamado \textbf{\texttt{fixxml.sh}},
situado en el directorio \texttt{misc/transformation} y dependiente de
\textbf{gawk} toma como argumento la ruta de un fichero a arreglar e imprime el
fichero correcto por la salida est�ndar.

Los errores que �ste script arregla son los siguientes:

\begin{itemize}
\item A�ade un nodo ra�z al fichero de colecci�n (\texttt{<collection>} y
\texttt{</collection>})

\item Finaliza correctamente las etiquetas de las secciones
\texttt{doctrain} y \texttt{doctest} de ficheros de colecci�n err�neos
(por ejemplo, los que comiencen por \texttt{<doctrain>} y acaben con
\texttt{</doc>}, se cambiar�a esta �ltima etiqueta por \texttt{</doctrain>})
\end{itemize}

Recuerde, MIEX \textbf{excluir� del procesamiento} a los ficheros que no cumplan
el \textit{schema} proporcionado. Si usted fuerza evitar la
comprobaci�n utilizando \textit{schema} y el fichero de colecci�n est� mal
formado, el \textit{unmarshaller} lanzar� una excepci�n no recuperable y
abortar� la ejecuci�n.

Puede obtener m�s informaci�n acerca del formato correcto de los ficheros de
entrada en la secci�n \ref{sec:schema} de este mismo libro.

\subsection{Integraci�n con FK}

FK\footnote{http://www.aic.uniovi.es/pir/Software.htm} es un entorno para el
desarrollo de aplicaciones modulares en  C++ basado en un modelo de plugins
jer�rquicos, extensibles y reutilizables. FK\cite{FK} ofrece al programador
servicios para cargar din�micamente plugins que tengan un cierto interfaz y
autom�ticamente comprueba, convierte y enlaza las instancias particulares
especificadas en tiempo de ejecuci�n. Este programa forma parte de un proyecto
fin de carrera desarrollado por Manuel Roberto Berdasco Mart�nez.

Como herramienta experimental, en el directorio \texttt{misc/fk} se distribuye
un \textit{wrapper} para poder invocar f�cilmente MIEX (y por consiguiente la
JVM\footnote{Java Virtual Machine})
desde FK.

Hasta el momento no existe ninguna manera de configurar el \textit{wrapper} en
tiempo de ejecuci�n por lo que probablemente tendr� que modificar alg�n
\textit{path} en el fichero fuente antes de compilar y ejecutar para que
funcione correctamente.

Finalmente, el autor de esta documentaci�n \textbf{no recomienda} lanzar MIEX
utilizando FK, utilice el sistema habitual para realizar las ejecuciones.

%\input{metodologia/.tex}
%\input{metodologia/.tex}
%\input{metodologia/.tex}



\input{manuales.tex}

\chapter{Conclusiones y l�neas de futuro}

\chapter{Conclusiones y l�neas de futuro}

\chapter{Conclusiones y l�neas de futuro}

\input{conclusiones/conclusiones.tex}

\input{conclusiones/futuro.tex}



\section{El futuro de MIEX}



\section{El futuro de MIEX}



\appendix


\chapter{Penn Treebank II Tags\label{sec:penntags}}

En el siguiente ap�ndice se explica en forma de tabla el significado de cada una
de las etiquetas que se puedan ir asignando durante el an�lisis de las
oraciones.
 

\begin{center}

\begin{longtable}{rcl}

% Cabecera principal
\hline \hline \\[-2ex]
   \multicolumn{1}{c}{\textbf{N�mero}} &
   \multicolumn{1}{c}{\textbf{Etiqueta}} &
   \multicolumn{1}{c}{\textbf{Descripci�n}} \\%[0.5ex] 
   \hline
   \\[-2.6ex]
\endfirsthead

% Cabecera de las otras partes de la tabla
\multicolumn{3}{c}{{Continuaci�n del \tablename} \thetable{}}\\[1.5ex]
  \hline \hline \\[-2ex]
  \multicolumn{1}{c}{\textbf{N�mero}} &
  \multicolumn{1}{c}{\textbf{Etiqueta}} &
  \multicolumn{1}{c}{\textbf{Descripci�n}} \\
  \\[-2.8ex]
\endhead

% Pi� de las p�ginas, excepto de la �ltima
  \multicolumn{3}{l}{{Contin�a en la siguiente p�gina\ldots}} \\
\endfoot

% Pi� de la �ltima p�gina
\endlastfoot

% Datos
1 & S & simple declarative clause, i.e. one that is not introduced \\
  &   & by a (possible empty) subordinating conjunction or a\\
  &   & \textit{wh}-word and that does not exhibit subject-verb inversion\\
\hline
2 & SBAR & Clause introduced by a (possibly empty) subordinating conjunction.\\
\hline
3 & SBARQ & Direct question introduced by a \textit{wh}-word or \\
  &       & a \textit{wh}-phrase.  Indirect questions and relative clauses\\
  &       & should be bracketed as SBAR, not SBARQ.\\
\hline  
4 & SINV & Inverted declarative sentence, i.e. one in which the\\
  &      & subject follows the tensed verb or modal.\\
\hline  
5 & SQ   & Inverted yes/no question, or main clause of a \textit{wh}-question,\\
  &      & following the \textit{wh}-phrase in SBARQ.\\
\hline \hline

\caption{Etiquetas a nivel de cl�usula}
\end{longtable}


%%%%%%%%%%%
%%%%%%%%%%%

\begin{longtable}{rcl}

% Cabecera principal
\hline \hline \\[-2ex]
   \multicolumn{1}{c}{\textbf{N�mero}} &
   \multicolumn{1}{c}{\textbf{Etiqueta}} &
   \multicolumn{1}{c}{\textbf{Descripci�n}} \\%[0.5ex] 
   \hline
   \\[-2.6ex]
\endfirsthead

% Cabecera de las otras partes de la tabla
\multicolumn{3}{c}{{Continuaci�n del \tablename} \thetable{}}\\[1.5ex]
  \hline \hline \\[-2ex]
  \multicolumn{1}{c}{\textbf{N�mero}} &
  \multicolumn{1}{c}{\textbf{Etiqueta}} &
  \multicolumn{1}{c}{\textbf{Descripci�n}} \\
  \\[-2.8ex]
\endhead

% Pi� de las p�ginas, excepto de la �ltima
  \multicolumn{3}{l}{{Contin�a en la siguiente p�gina\ldots}} \\
\endfoot

% Pi� de la �ltima p�gina
\endlastfoot

% Datos
1 & ADJP & Adjective Phrase\\
\hline
2 & ADVP & Adverb Phrase\\
\hline
3 & CONJP & Conjunction Phrase\\
\hline
4 & FRAG & Fragment\\
\hline
5 & INTJ & Interjection. Corresponds approximately to the part-of-speech\\
  &      & tag UH\\
\hline
6 & LST & List marker. Includes surrounding punctuation\\
\hline
7 & NAC & Not a Constituent; used to show the scope of certain prenominal\\
  &     & modifiers within an NP\\
\hline
8 & NP & Noun Phrase\\
\hline
9 & NX & Used within certain complex NPs to mark the head of the NP\\
\hline
10 & PP & Prepositional Phrase\\
\hline
11 & PRN & Parenthetical\\
\hline
12 & PRT & Particle. Category for words that should be tagged RP\\
\hline
13 & QP & Quantifier Phrase\\
\hline
14 & RRC & Reduced Relative Clause\\
\hline
15 & UCP & Unlike Coordinated Phrase\\
\hline
16 & VP & Vereb Phrase\\
\hline
17 & WHADJP & \textit{Wh}-adjective Phrase\\
\hline
18 & WHAVP & \textit{Wh}-adverb Phrase\\
\hline
19 & WHNP & \textit{Wh}-noun Phrase\\
\hline
20 & WHPP & \textit{Wh}-prepositional Phrase.\\
\hline
21 & X & Unknown, uncertain, or unbracketable\\
\hline \hline

\caption{Etiquetas a nivel de frase}

\end{longtable}


%%%%%%%%%%%%%
%%%%%%%%%%%%%


\begin{longtable}{rcl}

% Cabecera principal
\hline \hline \\[-2ex]
   \multicolumn{1}{c}{\textbf{N�mero}} &
   \multicolumn{1}{c}{\textbf{Etiqueta}} &
   \multicolumn{1}{c}{\textbf{Descripci�n}} \\%[0.5ex] 
   \hline
   \\[-2.6ex]
\endfirsthead

% Cabecera de las otras partes de la tabla
\multicolumn{3}{c}{{Continuaci�n del \tablename} \thetable{}}\\[1.5ex]
  \hline \hline \\[-2ex]
  \multicolumn{1}{c}{\textbf{N�mero}} &
  \multicolumn{1}{c}{\textbf{Etiqueta}} &
  \multicolumn{1}{c}{\textbf{Descripci�n}} \\
  \\[-2.8ex]
\endhead

% Pi� de las p�ginas, excepto de la �ltima
  \multicolumn{3}{l}{{Contin�a en la siguiente p�gina\ldots}} \\
\endfoot

% Pi� de la �ltima p�gina
\endlastfoot

% Datos
1 & CC & Coordinating conjunction\\
\hline
2 & CD & Cardinal number\\
\hline
3 & CC & Coordinating conjunction\\
\hline
4 & CD & Cardinal number\\
\hline
5 & DT & Determiner\\
\hline
6 & EX & Existential there\\
\hline
7 & FW & Foreign word\\
\hline
8 & IN & Preposition or subordinating conjunction\\
\hline
9 & JJ & Adjective\\
\hline
10 & JJR & Adjective, comparative\\
\hline
11 & JJS & Adjective, superlative\\
\hline
12 & LS & List item marker\\
\hline
13 & MD & Modal\\
\hline
14 & NN & Noun, singular or mass\\
\hline
15 & NNS & Noun, plural\\
\hline
16 & NNP & Proper noun, singular\\
\hline
17 & NNPS & Proper noun, plural\\
\hline
18 & PDT & Predeterminer\\
\hline
19 & POS & Possessive ending\\
\hline
20 & PRP & Personal pronoun\\
\hline
21 & PRP\$ & Possessive pronoun (prolog version PRP-S)\\
\hline
22 & RB & Adverb\\
\hline
23 & RBR & Adverb, comparative\\
\hline
24 & RBS & Adverb, superlative\\
\hline
25 & RP & Particle\\
\hline
26 & SYM & Symbol\\
\hline
27 & TO & to\\
\hline
28 & UH & Interjection\\
\hline
29 & VB & Verb, base form\\
\hline
30 & VBD & Verb, past tense\\
\hline
31 & VBG & Verb, gerund or present participle\\
\hline
32 & VBN & Verb, past participle\\
\hline
33 & VBP & Verb, non-3rd person singular present\\
\hline
34 & VBZ & Verb, 3rd person singular present\\
\hline
35 & WDT & Wh-determiner\\
\hline
36 & WP & Wh-pronoun\\
\hline
37 & WP\$ & Possessive wh-pronoun (prolog version WP-S)\\
\hline
38 & WRB & Wh-adverb\\
\hline \hline

\caption{Etiquetas a nivel de palabra}

\end{longtable}
\end{center}


%\chapter{C�digo fuente\label{sec:source}} 

La versi�n actual del c�digo (versi�n \codeversion) se incluye en este
ep�grafe. De cara a poner el software en producci�n se aconseja obtener la
�ltima versi�n de la p�gina web del proyecto\footnote{http://miex.sf.net}

\section*{Paquete Config}

\subsection*{ConfigCMDLine.java}

\lstinputlisting[language=Java]{\sourcepath/config/ConfigCMDLine.java}

\subsection*{ConfigFile.java}

\lstinputlisting[language=Java]{\sourcepath/config/ConfigFile.java}

\section*{Paquete Datastr}

\subsection*{ExtendedTaggedWord.java}
\lstinputlisting[language=Java]{\sourcepath/datastr/ExtendedTaggedWord.java}

\subsection*{MyCategories.java}
\lstinputlisting[language=Java]{\sourcepath/datastr/MyCategories.java}

\subsection*{MyCollection.java}
\lstinputlisting[language=Java]{\sourcepath/datastr/MyCollection.java}

\subsection*{MyDoc.java}
\lstinputlisting[language=Java]{\sourcepath/datastr/MyDoc.java}

\section*{Paquete filter}

\subsection*{GlobalFilter.java}
\lstinputlisting[language=Java]{\sourcepath/filter/GlobalFilter.java}

\subsection*{NumbersDetector.java}
\lstinputlisting[language=Java]{\sourcepath/filter/NumbersDetector.java}

\subsection*{Porter.java}
\lstinputlisting[language=Java]{\sourcepath/filter/Porter.java}

\subsection*{StopWordsDetector.java}
\lstinputlisting[language=Java]{\sourcepath/filter/StopWordsDetector.java}

\subsection*{UselessTagsDetector.java}
\lstinputlisting[language=Java]{\sourcepath/filter/UselessTagsDetector.java}

\section*{Paquete Input}

\subsection*{FieldsParser.java}
\lstinputlisting[language=Java]{\sourcepath/input/FieldsParser.java}

\subsection*{SAXCollectionUnmarshaller.java}
\lstinputlisting[language=Java]{\sourcepath/input/SAXCollectionUnmarshaller.java
}

\subsection*{XMLValidator.java}
\lstinputlisting[language=Java]{\sourcepath/input/XMLValidator.java}

\section*{Paquete Run}

\subsection*{Miex.java}
\lstinputlisting[language=Java]{\sourcepath/run/Miex.java}

\section*{Paquete Semantic}

\subsection*{Extractor.java}
\lstinputlisting[language=Java]{\sourcepath/semantic/Extractor.java}

\section*{Paquete Sql}

\subsection*{SQLHandler.java}
\lstinputlisting[language=Java]{\sourcepath/sql/SQLHandler.java}

\section*{Paquete Tools}

\subsection*{HexString.java}
\lstinputlisting[language=Java]{\sourcepath/tools/HexString.java}

\subsection*{MD5.java}
\lstinputlisting[language=Java]{\sourcepath/tools/MD5.java}

\section*{Otros}

\subsection*{Makefile (MIEX)}
\lstinputlisting{\trunkpath/Makefile}

\subsection*{Build.xml (MIEX)}
\lstinputlisting{\trunkpath/build.xml}

\subsection*{Makefile (Documentaci�n)}
\lstinputlisting{Makefile}


\input{anexos/licencias.tex}

\chapter{Referencias}

% A�adir todas las footnotes

\begin{itemize}
 \item MIEX - Metadata and
Information Extractor from small XML documents
 <\url{http://swaml.berlios.de/}>
\end{itemize}



%\bibliography{bibliografia}

\end{document}

