%%%%%%%%%%%%%%%%%%%%%%%%%%%%%%%%%%%%%%%%%%%%%%%%%%%%%%%%%%%%%%%
%																															%
%	MIEX																											  %
% Extracci�n de informaci�n de documentos XML mediante SP     %
%																															%
% Proyecto Fin de Carrera (EUITITG)                           %
% nn de Septiembre de 2007 - Gij�n                            %
%																															%
% Licencia GNU GPL2                                           %
%																															%
% Ignacio Barrientos <nacho@criptonita.com>										%
%																															%
%%%%%%%%%%%%%%%%%%%%%%%%%%%%%%%%%%%%%%%%%%%%%%%%%%%%%%%%%%%%%%%

%%%%%%%%%%%%%%%%%%%%%%%%%%%%%%%%%%%%%%%%%%%%%%%%%%%%%%%%%%%%%%%%
%%%%%%%%%%%%%%%%%% Configuraci�n del entorno %%%%%%%%%%%%%%%%%%%
%%%%%%%%%%%%%%%%%%%%%%%%%%%%%%%%%%%%%%%%%%%%%%%%%%%%%%%%%%%%%%%%

% Tipo de documento LaTeX-beamer
\documentclass[14pt]{beamer}

% Acentos pasivos
\usepackage[latin1]{inputenc}
\usepackage[T1]{fontenc}
\usepackage{ulem}

% Look and feel
\usetheme{Copenhagen}
\usecolortheme{rose}
%\usefonttheme{structurebold}

% Color de fondo por defecto.
%\beamertemplatesolidbackgroundcolor{yellow!30!white}

% Informaci�n de autor�a
\title{MIEX}
\subtitle{Extracci�n de informaci�n de documentos XML mediante Stanford Parser}
\date{nn de Septiembre de 2007}
\institute
{
	\bf
	{
		Proyecto Fin de Carrera\\
		E.U.I.T. Inform�tica y Telem�tica de Gij�n
	}
}
\author{Ignacio Barrientos <nacho@debian.org>}


%%%%%%%%%%%%%%%%%%%%%%%%%%%%%%%%%%%%%%%%%%%%%%%%%%%%%%%%%%%%%%%%
%%%%%%%%%%%%%%%%%% Inicio de la presentaci�n %%%%%%%%%%%%%%%%%%%
%%%%%%%%%%%%%%%%%%%%%%%%%%%%%%%%%%%%%%%%%%%%%%%%%%%%%%%%%%%%%%%%
\begin{document}

%-------------------------- SLIDE ---------------------------%

\frame
{
	% Portada
	\titlepage
}

%-------------------------- SLIDE ---------------------------%

\frame
{
	% Solo interesan las secciones para el �ndice
	\setcounter{tocdepth}{1}
	% Creamos tabla de contenidos
	\tableofcontents
}

%%%%%%%%%%%%%%%%%%%%%%%%%%%%%%%%%%%%%%%%%%%%%%%%%%%%%%%%%%%%%%%%%%%%%%%%%
%	Introducci�n																													%
%%%%%%%%%%%%%%%%%%%%%%%%%%%%%%%%%%%%%%%%%%%%%%%%%%%%%%%%%%%%%%%%%%%%%%%%%

\section{Introducci�n}

%------------------------ DEMO SLIDE ------------------------%

\frame
{
	\frametitle{Why Inforg\footnote{http://www.inforg.uniovi.es} could be better}

	\begin{block}{Students behaviour}
		\begin{itemize}
			% S�lo visible en la primera y segunda parte de la slide
			\item<1-2> Sane
			% De la segunda en adelante, con alerta de la segunda en
			% adelante.
			\item<2- | alert@2-> Really?
			% De la tercera en adelante
			\item<3-> Yes
		\end{itemize}
	\end{block}
}

%-------------------------- SLIDE ---------------------------%

\frame
{
  \frametitle{Desventajas del analizador}

    \begin{itemize}
      \item<1-> Deficiente sistema de entradas (\sout{batch})
			\uncover{\item<2-> Manejo \textit{poco amigable}}
			\item<3-> El almacenamiento no es \textbf{flexible}
    \end{itemize}
}

%-------------------------- SLIDE ---------------------------%

\frame
{
  \frametitle{La cara amigable del Stanford Parser}

  \begin{figure}[h]
    \begin{center}
      \includegraphics[scale=0.3]{images/spui.png}
    \end{center}
  \end{figure}
}

%%%%%%%%%%%%%%%%%%%%%%%%%%%%%%%%%%%%%%%%%%%%%%%%%%%%%%%%%%%%%%%%%%%%%%%%%
% Funcionamiento                                                        %
%%%%%%%%%%%%%%%%%%%%%%%%%%%%%%%%%%%%%%%%%%%%%%%%%%%%%%%%%%%%%%%%%%%%%%%%%

\section{Funcionamiento}

%-------------------------- SLIDE ---------------------------%

\frame
{
  \frametitle{�C�mo funciona?}

  \begin{figure}[h]
    \begin{center}
      \includegraphics[scale=0.8]{images/miex.png}
    \end{center}
  \end{figure}
}

%-------------------------- SLIDE ---------------------------%

\frame
{
  \frametitle{�M�s detalle, por favor!}

	\begin{figure}[h]
		\begin{center}
			\includegraphics[scale=0.1425]{images/build/miex-libs.png}
		\end{center}
	\end{figure}
}

%%%%%%%%%%%%%%%%%%%%%%%%%%%%%%%%%%%%%%%%%%%%%%%%%%%%%%%%%%%%%%%%%%%%%%%%%
% Demo + Preguntas                                                      %
%%%%%%%%%%%%%%%%%%%%%%%%%%%%%%%%%%%%%%%%%%%%%%%%%%%%%%%%%%%%%%%%%%%%%%%%%

\section{Demo + Preguntas}

%-------------------------- SLIDE ---------------------------%

\frame
{
	\frametitle{�Preguntas?}

	\begin{beamerboxesrounded}[shadow=true]{Autor�a}

	\begin{center}

	\normalsize{Ignacio Barrientos Arias}\\
	\small{nacho@debian.org}\\

	\end{center}

	\end{beamerboxesrounded}

	%%%%

	\begin{beamerboxesrounded}[shadow=true]{Licencia}

	\begin{center}

	\small
	{
	GNU General Public License (version 2)\\
	Sources at: http://miex.sf.net/
	}

	\end{center}

	\end{beamerboxesrounded}

	%%%%

	\begin{beamerboxesrounded}[shadow=true]{Tecnolog�a}

	\begin{center}

	\small
	{
	\LaTeX{} Beamer \scriptsize{http://latex-beamer.sf.net/}
	}

	\end{center}

	\end{beamerboxesrounded}

}

% That's all folks
\end{document}
